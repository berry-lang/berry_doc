\chapter*{序\quad 言}

\pagestyle{empty}
\thispagestyle{empty}

几年以前我曾尝试过将Lua脚本语言移植到STM32F4单片机,后来我在ESP8266上体验过一款基于Lua的固件:NodeMCU,这些经历使我体验到了使用脚本开发的便利。后来我又接触了几款脚本语言,例如Python、JavaScript、Basic以及MATLAB等,在这些语言中只有少数是适合移植到单片机平台的。当时我比较关注Lua,因为它是一款定位非常精简的嵌入式脚本语言,其设计目标就是嵌入到宿主程序中使用。然而对于单片机而言Lua依然大了一点,我无法在存储器比较小的32位单片机上运行它。为此,我开始阅读Lua代码并在此基础上开发自己的脚本语言------Berry。

Berry是一款超轻量级的嵌入式脚本语言,它还是一款多范式的动态语言。支持面向对象、面向过程和函数式(支持比较少)风格。

Berry在很多地方都参考了Lua,但是它完全面向性能较低的嵌入式系统,这些系统可能只有几KB的RAM和不到64KB的ROM并且可能没有硬件浮点支持(即便支持通常也是单精度浮点而没有双精度浮点),因此要运行一个功能全面的脚本语言十分困难。为此,Berry不会试图去提供复杂的语法糖,而只是实现比较比较重要的功能。在精简设计的同时我尽可能满足功能的需求,使Berry成为一个功能全面的脚本语言。由于有Lua解释器作为模板,我的工作简单了很多,Berry的编译器和字节码参考了Lua的实现,这是一种对存储资源要求很低但是很高效的设计。

Berry解释器使用一个一趟式编译器把源代码编译成字节码,这种编译器不需要生成抽象语法树(AST)并且只需要读取一遍源代码就可以完成字节码的编译工作。字节码类似于机器语言,不过运行它的不是物理机器而是Berry虚拟机(VM)。Berry使用基于寄存器的虚拟机(还有一种基于堆栈的虚拟机),一般认为寄存器式的虚拟机要比堆栈式的虚拟机需要更少的字节码并且性能更好。当然,编译器和虚拟机对Berry源代码而言是透明的,本质上来说,一门语言并不依赖于解释器或者编译器的具体实现方式。

我是直到后来才了解到MicroPython项目,一个为单片机设计的Python解释器,该解释器十分精简,使得Python可以在单片机上运行。如今Python十分火热,而这款为单片机实现的Python解释器也非常受欢迎。而我没有使用现有语言的语法,这可能会影响到Berry的使用。不过Berry的语法十分简单,如果你有其他语言的基础,可能只要几个小时就能掌握。如果你需要移植Berry的解释器,需要保证你使用的编译器提供对C99标准的支持(我先前完全遵守C89,但是后来的一些优化工作使我放弃了这个决定),ARM处理器开发中常见ARMCC(KEIL MDK)、ICC(IAR)以及GCC等编译器都支持C99。

本文档介绍Berry的语法规则、标准库等设施,最后会指导读者去移植并扩展Berry。
