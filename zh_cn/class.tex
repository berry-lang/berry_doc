\chapter{面向对象功能}

出于优化方面的考虑,Berry没有将简单类型作为对象,这些简单类型包括 \texttt{nil} 类型、数值类型、布尔类型和字符串类型。但是Berry提供了类来实现对象机制,在Berry的基本数据类型中,\texttt{list}、\texttt{map} 和 \texttt{range} 是类对象。一个对象是是包含数据和方法的集合,其中数据由一些变量来构成,而方法则是函数。对象的类型称为类(class),而对象的实体称为实例(instance)。

\section{类和实例}

\subsection{类的声明}

要使用一个类首先要进行声明。类的声明由关键字 \texttt{class} 开始,声明中要指定类的成员变量和方法,这是声明类的一个例子:
\begin{lstlisting}[language=berry, numbers=none]
class person
    var name, age
    def init(name, age)
        self.name = name
        self.age = age
    end
    def tostring()
        return 'name: ' + str(self.name) + ', age: ' + str(self.age)
    end
end
\end{lstlisting}

类的成员变量使用关键字 \texttt{var} 声明,而成员方法使用关键字 \texttt{def} 声明。目前Berry不支持在定义时初始化成员变量,因此成员变量的初始化工作应该由构造函数来完成。类的属性不能再声明完成后再做修改,因此类是一种只读对象\footnote{这种设计是为了保证在实现解释器的时候可以在C语言中静态构造类并使用 \texttt{const} 属性修饰以节省RAM}。

Berry 的类不支持访问限制,类的所有属性都对外部可见。在原生类中可以使用一些技巧使属性对 Berry 代码不可见(通常是让成员名字以``\text{.}''开头)。可以使用一些约定来限制对类中成员的访问,比如约定使用下划线开头的属性是私有属性,这种约定并不会在语法层面上有什么用,但是有利于代码的逻辑结构。

\subsection{实例化}

类本身只是一种抽象的描述。以汽车为例,我知道汽车的概念,而当我们真的要使用汽车的时候则需要真实的汽车。使用类的情况也类似,我们不会仅仅去使用这种抽象的描述,而是需要根据这种描述去生产出一个具体的对象。这个过程叫做\textbf{类的实例化},简称实例化,实例化产生的具体对象称为\textbf{实例}。类本身不具有数据,而实例化根据类所描述的信息生产一个实例并赋予实例具体的数据。

\subsection{方法和 \texttt{self} 参数}

类的方法本质上也是函数,与普通的函数不同,方法会隐式地传入一个 \texttt{self} 参数,且 \text{self} 总是作为第一个参数,该参数存储当前实例的引用。由于 \texttt{self} 参数的存在,方法的参数数量会比声明时定义的参数数量多一个。这里我们用一个简单的例子演示:
\begin{lstlisting}[language=berry, numbers=none]
class Test
    def method()
        return self
    end
end
object = Test()
print(object)
print(object.method())
\end{lstlisting}
这个例子中定义了一个 \texttt{Test} 类,它有一个 \texttt{method} 方法,该方法返回它的 \texttt{self} 参数。例程中的最后两行分别打印了 \texttt{Test} 类的实例 \texttt{object} 的值和使用 \texttt{method} 方法的返回值。该例子的运行结果为\footnote{由于实例对象是动态分配的,它们的内存地址是随机的,读者运行这段代码的结果可能与此处不同。}
\begin{lstlisting}[numbers=none]
<instance: 00E880D4>
<instance: 00E880D4>
\end{lstlisting}
可以看出,方法的 \texttt{self} 参数和使用实例的名字(例子中的 \texttt{object})都是表示同一个对象,它们都是实例的引用。使用 \texttt{self} 可以在方法中访问实例的成员或者属性。

\subsection{构造函数和析构函数}

\subsubsection{构造函数}

类的构造函数为 \texttt{init} 方法,构造函数会在类实例化的时候调用,因此构造函数一般用于成员的初始化工作,例如:
\begin{lstlisting}[language=berry, numbers=none]
class Test
    var a
    def init()
        self.a = 'this is a test'
    end
end
\end{lstlisting}
这个例子中的构造函数将 \texttt{Test} 类的 \texttt{a} 成员初始化为字符串 \texttt{'this is a test'}。如果我们实例化该类,就可以获得成员 \texttt{a} 的值:
\begin{lstlisting}[language=berry, numbers=none]
print(Test().a) # this is a test
\end{lstlisting}

\subsubsection{析构函数}

类的析构函数为 \texttt{deinit} 方法,析构函数会在实例被销毁时调用,析构函数一般用于完成一些清理工作。由于垃圾回收机制会自动释放无用对象的内存,因此不需要在析构函数中释放内存(也没有办法在析构函数中释放内存)。在大部分情况下都不需要使用析构函数,除非某个类要求在销毁时必须进行一定的处理,一个典型的例子是文件对象在销毁时必须关闭文件。

\section{类的继承}

Berry 只支持单继承,也就是类只能有一个基类,基类使用运算符 \texttt{:} 来声明:
\begin{lstlisting}[language=berry, numbers=none]
class Test : Base
    ...
end
\end{lstlisting}
这里 \texttt{Test} 类继承自 \texttt{Base} 类。子类会继承基类的所有方法和属性,同时你可以在子类中覆盖它们,这个机制被称为\textbf{重载}。通常情况下,我们只会重载方法,而不必重载属性。

Berry 类的继承机制比较简单,子类会包含基类的引用,实例对象也是类似。在实例化一个有基类的类时其实会生成多个对象,这些对象会根据继承关系链在一起,最后我们会拿到继承链最末端的实例对象。

\section{方法重载}

\textbf{重载}是指子类和基类使用同名的方法,而子类的方法将会覆盖基类方法的机制。准确地说成员变量也可以重载,但是这种重载没有任何意义。方法的重载分为普通方法重载以及运算符重载。

\subsection{普通方法重载}

\subsection{运算符重载}

\section{访问基类对象}
