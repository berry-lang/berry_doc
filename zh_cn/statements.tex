\chapter{控制语句}

很多编程语言都提供一些语句使程序实现分支结构和循环结构,Berry也提供这样的语句,在这里,这类语句被称为控制语句。Berry的控制语句主要分为三类:条件语句,迭代语句和跳转语句。程序流程图很容易翻译成这几种语句,反之亦然。

\section{条件语句}

Berry提供\texttt{if}语句来实现条件控制执行的功能。\texttt{if}语句会根据条件表达式的真(\texttt{true})或假(\texttt{false})来控制程序执行的分支。在一些语言中,实现条件控制还有其他的选择,例如C、C++等语言提供\texttt{switch}语句,不过为了简化设计,Berry没有支持\texttt{switch}语句。

\subsection{\texttt{if}语句}

\textbf{\texttt{if}语句}可以用来实现简单的分支结构,它根据某种判断条件的真假来选择程序运行的分支。\texttt{if}语句的形式为

\begin{algorithm}
    \texttt{if (} $\bm{condition}$ \texttt{)} \\
    \qquad $\bm{block}$ \\
    \texttt{end}
\end{algorithm}

$\bm{condition}$是条件表达式,当$\bm{condition}$的值为\texttt{true}时,将会执行第二行的$\bm{block}$,否则会跳过该$\bm{block}$并执行\texttt{end}后面的语句。在$\bm{block}$被执行的情况下,块中的最后一条语句执行完以后会离开\texttt{if}语句并开始执行\texttt{end}后面的语句。

如果希望条件满足和不满足时都有对应的分支可供执行则需要使用\textbf{\texttt{if else}语句}。该语句的形式为
\begin{algorithm}
    \texttt{if (} $\bm{condition}$ \texttt{)} \\
        \qquad $\bm{block}$ \\
    \texttt{else} \\
        \qquad $\bm{block}$ \\
    \texttt{end}
\end{algorithm}

与简单的\texttt{if}语句不同的是,\texttt{if else}语句在$\bm{condition}$的值为\texttt{false}时将会执行\texttt{else}分支下的$\bm{block}$。无论哪个分支的$\bm{block}$被执行,在其执行完以后都会跳出\texttt{if else}语句——执行\texttt{end}后面的语句,也就是说,无论$\bm{condition}$的值是\texttt{true}还是\texttt{false},都只有一个$\bm{block}$被执行。

有时候我们需要嵌套使用\texttt{if}语句,一种情况是要在\texttt{else}分支下嵌套一个\texttt{if}语句:
\begin{lstlisting}[language=berry, numbers=none]
if (expr)
    block
else
    if (expr)
        block
    end
end
\end{lstlisting}
很明显,这种写法会增加代码的缩进层次,并且结尾处要使用多个\texttt{end},比较繁琐。作为改进,Berry提供了\texttt{elif}分支来优化上面的写法。使用\texttt{elif}分支与上述代码等价,其形式为
\begin{algorithm}
    \texttt{if (} $\bm{condition}$ \texttt{)} \\
        \qquad $\bm{block}$ \\
    \texttt{elif (} $\bm{condition}$ \texttt{)} \\
        \qquad $\bm{block}$ \\
    \texttt{else} \\
    \qquad $\bm{block}$ \\
    \texttt{end}
\end{algorithm}

\texttt{elif}分支必须用在\texttt{if}分支之后并可以连续使用多次。如果\texttt{elif}分支对应的$\bm{condition}$满足则会执行分支下的$\bm{block}$。

\section{迭代语句}

迭代语句又叫循环语句,用于重复执行某种操作直到终止条件满足。Berry提供\texttt{while}语句和\texttt{for}两种迭代语句。很多语言也是提供这两种语句用于迭代。Berry的\texttt{while}语句和C/C++中的\texttt{while}语句类似,而Berry的\texttt{for}语句仅用于遍历容器中的元素,类似一些语言提供的\texttt{foreach}语句,以及C++ 11引入的新\texttt{for}语句风格。而C风格的\texttt{for}语句在Berry中则没有实现。

\subsection{\texttt{while}语句}

\textbf{\texttt{while}语句}是一种基本的迭代语句,\texttt{while}语句根据一个判断条件来进入循环并在条件不满足时终止循环,其模式为
\begin{algorithm}
    \texttt{while (} $\bm{condition}$ \texttt{)} \\
        \qquad $\bm{block}$ \\
    \texttt{end}
\end{algorithm}

程序运行到\texttt{while}语句时会检测$\bm{condition}$表达式的真假,如果为真则执行$\bm{block}$。执行完$\bm{block}$中的最后一条语句后,程序会跳转到\texttt{while}语句的开始位置并重新检测$\bm{condition}$的真假,为真则开始下一轮的循环。当$\bm{condition}$表达式为假时\texttt{while}语句将会终止,此时会继续执行\texttt{end}后的第一条语句。显然,如果$\bm{condition}$表达式在第一次判断时为假则不会执行$\bm{block}$(和\texttt{if}语句的$\bm{condition}$表达式为假时一样)。

\subsection{\texttt{for}语句}

Berry的\textbf{\texttt{for}语句}用于遍历容器中的元素,其形式为
\begin{algorithm}
    \texttt{for (} $\bm{varaible}$ \texttt{:} $\bm{expression}$ \texttt{)} \\
        \qquad $\bm{block}$ \\
    \texttt{end}
\end{algorithm}

$\bm{expression}$表达式的值必须是一个可迭代的容器,例如\texttt{range}类。\texttt{for}语句会从容器中获取一个迭代器,通过对调用迭代器的调用实现每次循环得到容器中的一个元素。

$\bm{varaible}$称为迭代变量,迭代变量总是在\texttt{for}语句中定义。因此$\bm{varaible}$必须是一个变量名而不能是表达式。每次循环时从迭代器中获取的容器元素将会赋值给迭代变量,该过程发生在$\bm{block}$的第一条语句之前。

\texttt{for}语句会检查迭代器中是否还有未访问的元素可供迭代,如果有则开始下一轮的迭代,否则会结束\texttt{for}语句并执行\texttt{end}后面的语句。目前,Berry只提供只读迭代器,也就是说无法通过\texttt{for}语句中的迭代变量来修改容器中的元素。

迭代变量$\bm{varaible}$的作用域仅限于循环体$\bm{block}$中,该变量不会和作用域外的同名变量有任何关系。为了说明这一点,我们通过一个例子来说明,在这个例子里,我们使用\texttt{for}语句来访问\texttt{rang}实例中的所有元素并打印出来。当然,我们还通过这个例子来演示循环变量的作用域。
\begin{lstlisting}[language=berry]
i = "Hi, I'm fine." # outer variable
for (i : 0 .. 2)
    print(i)        # iteration variable
end
print(i)
\end{lstlisting}

在这个例子中,相对于第2行定义的的迭代变量\texttt{i}而言,第1行定义的变量\texttt{i}为外部变量。运行这个例子将得到下面的结果
\begin{lstlisting}[numbers=none]
0
1
2
Hi, I'm fine
\end{lstlisting}
可以看出,迭代变量\texttt{i}和外部变量\texttt{i}是两个不同的变量,它们只是名字相同,但是作用域不同。

\section{跳转语句}
