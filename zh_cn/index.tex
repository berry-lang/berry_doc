\documentclass[UTF8, oneside]{book}
    \usepackage{geometry}
    \usepackage{ctex}
    \usepackage{fancyhdr}
    \usepackage{makecell}
    \usepackage{multirow}
    \usepackage{listings}
    \usepackage{tikz}
    \usepackage{fontspec, xunicode, xltxtra}
    \usepackage{titlesec}
    \usepackage{float}
    \usepackage{caption}
    \usepackage{amsmath, amssymb}
    \usepackage[colorlinks, linkcolor=black, anchorcolor=black, citecolor=black]{hyperref}
    \usepackage[justification=centering]{caption}
    \usepackage{subcaption}
    \usepackage{color}
    \usepackage[super,square]{natbib}
    \usepackage{blindtext}
    \usepackage{bm}
    \usepackage{enumitem}
    \usepackage{graphicx}
    \usepackage{algorithm2e}
    \usepackage{ebgaramond}

    % 设置字体
    \setCJKmainfont[BoldFont=SimHei]{NSimSun}
    \setCJKmonofont{NSimSun}
    % 定义字体
    \setCJKfamilyfont{fzsh}{FZSongHei-B07S} % 方正宋黑
    \newcommand{\songhei}{\CJKfamily{fzsh}}
    % 定义字号
    \newcommand{\chuhao}{\fontsize{42pt}{54.6pt}\selectfont}
    \newcommand{\sanhao}{\fontsize{15.75pt}{20.5}\selectfont}

    % 页边距设置
    \geometry{left=3.18cm, right=3.18cm, top=2.54cm, bottom=2.54cm}

    % 设置标题和注解格式
    \titleformat{\chapter}{\centering\huge\bf\heiti}{\chinese{chapter}、}{0em}{}
    \titleformat{\section}{\Large\bf\heiti}{\thesection}{0.75em}{}
    \titleformat{\subsection}{\large\bf\heiti}{\thesubsection}{0.75em}{}
    \titlespacing*{\chapter} {0pt}{0pt}{20pt}
    %\captionsetup[figure]{name={图},labelsep=quad}
    \captionsetup[table]{name={表},labelsep=quad}
    % 标题样式重定义
    \renewcommand{\contentsname}{\centerline{\huge\bf\heiti 目\quad 录}}
    \renewcommand\bibname{\centerline{\normalsize\bf\heiti [参考文献]}}

    % 标题与上下文间隔
    \titlespacing*{\section}{0pt}{3ex}{1ex}
    \titlespacing*{\subsection}{0pt}{2ex}{1ex}
    \titlespacing*{\subsubsection}{0pt}{1ex}{0.5ex}

    % 列表行间距
    \setenumerate[1]{itemsep=1.5pt,partopsep=0pt,parsep=\parskip,topsep=5pt}
    \setitemize[1]{itemsep=1.5pt,partopsep=0pt,parsep=\parskip,topsep=5pt}
    \setdescription{itemsep=1.5pt,partopsep=0pt,parsep=\parskip,topsep=5pt}

    % 设置tikz
    \usetikzlibrary{arrows, automata}

    % 定义 PrintStyle 时使用打印样式
    % \def \PrintStyle {}

    % 设置代码样式
\ifx \PrintStyle \undefined
    \lstset{
        commentstyle=\small\color{green!50!black}\itshape,
        keywordstyle=\small\color{cyan!50!black}\bfseries,
        stringstyle=\small\color{purple},
        numberstyle=\small\ttfamily\color{gray},
    }
\else   % 打印用样式
    \lstset{
        commentstyle=\small\itshape,
        keywordstyle=\small\bfseries,
        numberstyle=\small\ttfamily,
    }
\fi
    % 通用代码样式
    \lstset{
        basicstyle=\small\ttfamily,
        columns=fullflexible,
        numbersep=1em,
        xleftmargin=\parindent,
        breakatwhitespace=false,
        numbers=left,
        captionpos=b,               % sets the caption-position to bottom
        breaklines=true,            % automatic line breaking only at whitespace
        keepspaces=true,
        showstringspaces=false,
        tabsize=4
    }

\begin{document}

    % 封面
    % 封面

%% temporary titles
% command to provide stretchy vertical space in proportion
\newcommand\nbvspace[1][3]{\vspace*{\stretch{#1}}}
% allow some slack to avoid under/overfull boxes
\newcommand\nbstretchyspace{\spaceskip0.5em plus 0.25em minus 0.25em}
% To improve spacing on titlepages
\newcommand{\nbtitlestretch}{\spaceskip0.6em}

% 长破折号
\newcommand{\cndash}{\raisebox{0.5mm}{------}}

\begin{titlepage}

    \title{\ebgaramond\Huge{\scshape The Berry Script Language\\Reference Manual}\\\Large{V0.1.0}}

    \author{官文亮}

    \maketitle

\end{titlepage}


    \chapter*{序\quad 言}

\pagestyle{empty}
\thispagestyle{empty}

几年前我曾尝试过将Lua脚本语言移植到STM32F4单片机,后来又在ESP8266上体验过一款基于Lua的固件:NodeMCU,这些经历使我体验到使用脚本开发的便利。后来我又接触了一些脚本语言,例如Python、JavaScript、Basic以及MATLAB等。目前看来,只有极少数语言是适合移植到单片机平台的。以前我比较关注Lua,因为它是一款定位非常精简的嵌入式脚本语言,其设计目标就是嵌入到宿主程序中使用。然而对于单片机而言,Lua解释器可能还不够小,我无法在存储器比较小的32位单片机上运行它。为此,我开始阅读Lua代码并在此基础上开发自己的脚本语言------Berry。

这是一款超轻量级的嵌入式脚本语言,它还是一款多范式的动态语言。支持面向对象、面向过程和函数式(支持比较少)风格。这款语言的很多方面都参考了Lua,不过它的语法还借鉴了其他一些语言的设计。如果读者已经了解一门高级语言,Berry的语法应该会非常容易掌握:该语言只有一些简单的规则,并且有着非常自然的作用域设计。

我考虑的主要应用场景是性能较低的嵌入式系统,这些系统的内存可能非常小,因此要运行一个功能全面的脚本语言十分困难。这意味着我们可能不得不做出取舍,Berry也许只会提供最常用而基础的核心功能,而其他的不必要的特性只作为可选的模块,这必然导致语言的标准库过小,甚至语言本身也会有不确定的设计(例如浮点数和整数的实现方式等)。这些折衷带来的收益则是较大的优化空间,而坏处显然是语言标准的不统一。

Berry的解释器参考了Lua解释器的实现,解释器主要分为编译器和虚拟机两部分。Berry的编译器是一种一趟式编译器生成字节码,这种方案不生成抽象语法树,因此比较节省内存。而虚拟机则是寄存器式的,一般来说寄存器式虚拟机比堆栈式虚拟机的效率要高一些。除了实现语言特性外,我们还希望优化解释器的内存占用和运行效率。目前,Berry解释器性能指标并不比主流的语言查,但是内存占用则非常小。

直到后来我才了解到MicroPython项目:一个为单片机设计的精简版Python解释器。如今Python十分火热,这款为单片机设计的Python解释器也非常受欢迎。与目前比较成熟的技术相比,Berry是一门新的语言,没有足够的用户基础,它的优势是语法易于掌握,以及可能在资源占用和性能方面占有优势。

如果你需要移植Berry解释器,需要保证你使用的编译器提供对C99标准的支持(我先前完全遵守C89,后来的一些优化工作使我放弃了这个决定)。目前大部分编译器都提供对对C99的支持,ARM处理器开发中常见ARMCC(KEIL MDK)、ICC(IAR)以及GCC等编译器也都支持C99。

本文档介绍Berry的语法规则、标准库等设施,最后会指导读者去移植并扩展Berry。本文档不阐述解释器的实现机制,有时间的话也许会在其他的文档中说明。

作者水平有限,加上行笔匆忙,文中如有纰漏或者错误,望读者不吝斧正。

\rightline{官文亮}

\rightline{2019年4月}


    \clearpage

    \setcounter{page}{1} % 目录页码
    \pagenumbering{roman} % 罗马数字页码
    \setcounter{tocdepth}{2} % 目录深度
    \tableofcontents % 目录

    \newpage
    \setcounter{page}{1}    % 重置页码计数
    \pagestyle{fancy}
    \pagenumbering{arabic}  % 数字页码

    \chapter{基本信息}

\section{开始使用}

你可以到Berry项目的GitHub页面(\url{https://github.com/gztss/berry})上获取Berry解释器的源代码。然后你需要编译Berry解释器,这些信息你可以到项目的README.md文档中查看。

在得到Berry解释器的可执行文件后(在Windows中文件名是``berry.exe'',在Linux和macOS中文件名是``berry''),你可以直接运行解释器\footnote{在Windows中你可以直接双击运行可执行文件,在Linux或者macOS中通常要使用``终端''(Terminal)来运行。你也可以在Windows的``命令提示符''(cmd)窗口中运行解释器。具体的使用方法请参考README.md文档。}并开始使用交互模式运行代码。

作为惯例,我们使用``Hello World''来入门。为了最快地达到目的,我们直接运行解释器,这时我们进入一个叫做``交互模式''(REPL)的界面,我们将看到如下信息:
\begin{lstlisting}[language=berry, numbers=none]
Berry 0.0.4 (build in Feb  1 2019, 13:14:04)
[GCC 8.1.0] on Windows (default)
>
\end{lstlisting}

这段文字中,前两行显示了Berry解释器的版本、编译时间、编译器和操作系统等信息,第三行的符号``\texttt{>}''叫做提示符,此时光标显示在提示符后面,我们可以直接在此处输入代码并在按下``Enter''键之后运行。为了完成我们的``Hello World''程序,在交互模式中输入``print('Hello World')''并执行,此时运行结果如下:
\begin{lstlisting}[language=berry, numbers=none]
Berry 0.0.4 (build in Feb  1 2019, 13:14:04)
[GCC 8.1.0] on Windows (default)
> print('Hello World')
Hello World
>
\end{lstlisting}

运行这行代码后解释器输出了一段文本``\texttt{Hello World}''。的原理是调用打印函数\texttt{print()}来输出字符串\texttt{'Hello World'}。在交互模式中,解释器会显示表达式的值(如果值不是\texttt{nil}的话),例如我们输入表达式\texttt{1 + 2}将会输出计算结果\texttt{3}:
\begin{lstlisting}[language=berry, numbers=none]
> 1 + 2
3
\end{lstlisting}

因此交互模式下最简单的``Hello World''程序是直接输入字符串\texttt{'Hello World'}并运行:
\begin{lstlisting}[language=berry, numbers=none]
> 'Hello World'
Hello World
\end{lstlisting}

你可以把Berry解释器的交互模式当成一个科学计算器来使用,不过,一些数学函数不能直接使用,而要先使用\texttt{import math}语句来导入数学库,然后在使用数学库中的函数时要使用``\texttt{math.}''作为前缀,例如\texttt{sin}函数要写成\texttt{math.sin}:
\begin{lstlisting}[language=berry, numbers=none]
> import math
> math.pi
3.14159
> math.sin(math.pi / 2)
1
> math.sqrt(2)
1.41421
\end{lstlisting}

\section{词法}

在介绍Berry的语法之前,我们先来看一段简单的代码(你可以在交互模式种运行这段代码):
\begin{lstlisting}[language=berry]
def func(x) # a function example
    return x + 1.5
end
print('func(10) =', func(10))
\end{lstlisting}

这段代码中定义了一个函数\texttt{func}并在后面调用了它。在了解这段代码怎样工作之前,我们先介绍Berry语言的语法元素。

以上代码中,语法元素的具体分类为:\texttt{def}、\texttt{return}和\texttt{end}是Berry语言的关键字;而第1行中的``\texttt{\# a function example}''被称为注释;\texttt{print}、\texttt{func}和\texttt{x}是一些标识符,它们通常用于表示一个变量;\texttt{1.5}和\texttt{10}这些数字被称为数值字面量,它们相当于日常生活中使用的数字;\texttt{'func(10) ='}是一个字符串字面量,他们大量用于需要表示文本的地方;\texttt{+}是一个加法运算符,这里使用加法运算符可以将变量\texttt{x}和数值\texttt{1.5}相加。

以上的分类实际上是从词法分析器的角度来做的。词法分析是Berry源代码解析的第一步,为了写出正确的源代码,我们先从最基础的词法开始介绍。

\subsection{注释}

注释是不会生成任何代码的一些文本,它们用于在源代码中做批注并给人们阅读,而编译器则不会解释它们的内容。Berry支持单行注释和跨行的块注释。单行注释从字符``\texttt{\#}'开始,直到换行字符结束。快注释从文本``\texttt{\#-}''开始,直到文本``\texttt{-\#}''结束。以下是使用注释的例子:
\begin{lstlisting}[language=berry, numbers=none]
# this is a line comment
#- this is a
   block comment
-#
\end{lstlisting}

和C语言类似,快注释不支持嵌套,以下代码将在第一个``\texttt{-\#}''文本处终止对注释的解析:
\begin{lstlisting}[language=berry, numbers=none]
#- some comments -# ... -#
\end{lstlisting}

\subsection{字面值}

字面值是编程时在源代码中直接写出的固定值。Berry的字面量有整数、实数、布尔量、字符串和nil。例如,数值\texttt{34}是一个整数字面值。

\subsubsection{数值字面值}

数值字面值包括\textbf{整数}(Integer)字面值和\textbf{实数}(Real)字面值。
\begin{lstlisting}[language=berry, numbers=none]
40      # integer
0x80    # hexadecimal literal (integer)
3.14    # real
1.1e-6  # real
\end{lstlisting}

数值字面值的写法和日常写法类似。Berry支持16进制的整数字面值,16进制字面值使用前缀\texttt{0x}或者\texttt{0X}开头,后面是一个16进制数。

\subsubsection{布尔字面值}

布尔值(Boolean)用来表示逻辑状态中的真和假。你可以使用\texttt{true}和\texttt{false}这两个关键字来表示布尔字面值。

\subsubsection{字符串字面值}

字符串(String)是一段文本,它的字面值写法是使用一对\texttt{'}或\texttt{"}包围字符串文本:
\begin{lstlisting}[language=berry, numbers=none]
'this is a string'
"this is a string"
\end{lstlisting}

\subsubsection{Nil字面值}

Nil表示空值,其字面值使用关键字\texttt{nil}来表示。

\subsection{标识符}

标识符(Identifier)是由用户定义的名字,它由下划线或者字母作为开头,再由若干个下划线、字母或者数字的组合构成。和大多数语言类似,Berry是大小写敏感的,因此标识符\texttt{A}和标识符\texttt{a}会解析为两种标识符。
\begin{lstlisting}[language=berry, numbers=none]
a
TestVariable
Test_Var
_init
baseCass
_
\end{lstlisting}

\subsection{关键字}

Berry保留以下记号作为语言的关键字:
\begin{lstlisting}[language=berry, numbers=none]
    if          elif        else        while       for         def
    end         class       break       continue    return      true
    false       nil         var         do          import      as
\end{lstlisting}

关键字的具体使用方法会在相关的章节中介绍。注意,不能将关键字作为标识符使用,由于Berry是大小写敏感的,因此\texttt{If}可以用于标识符。

\subsection{运算符}

Berry支持的运算符如表\ref{tab::operator_list}所示。

\begin{table}[htb]
    \centering
    \setlength{\tabcolsep}{4mm}
    \begin{tabular}{cclc} \Xhline{1pt}
        \makecell[c]{\textbf{优先级}} & \makecell[c]{\textbf{运算符}} & \makecell[c]{\textbf{说明}} & \makecell[c]{\textbf{结合性}} \\ \Xhline{1pt}
        1 & \texttt{()} & 分组符号 & - \\
        2 & \texttt{() [] .} & 函数调用,下标运算,域运算 & 左 \\
        3 & \texttt{+ - !} & 正号,负号,逻辑非 & 左 \\
        4 & \texttt{* / \%} & 乘法,除法,取余数 & 左 \\
        5 & \texttt{+ -} & 加法,减法 & 左 \\
        6 & \texttt{..} & 范围运算符 & 左 \\
        7 & \texttt{< <= == != > >=} & 关系运算符 & 左 \\
        8 & \texttt{\&\&} & 逻辑与 & 左 \\
        9 & \texttt{||} & 逻辑或 & 左 \\
        10 & \texttt{=} & 赋值 & 右 \\
        \Xhline{1pt}
    \end{tabular}
    \caption{运算符列表}
    \label{tab::operator_list}
\end{table}

\section{类型和值}

数据\textbf{类型}是数据的一种属性,它定义了数据的含义以及可以对数据进行的运算类型。

\subsection{内建类型}

Berry支持的基本类型有:

\begin{itemize}
    \item \textbf{Nil}:表示对象具有一个空值。
    \item \textbf{Integer}:有符号的整数。它的长度取决于具体的实现。例如,在32位平台中通常是一个32位的有符号整数。
    \item \textbf{Real}:实数。具体实现可以选择选用单精度浮点数或者双精度浮点数。
    \item \textbf{Boolean}:值可以是\texttt{true}或者\texttt{false}。
    \item \textbf{String}:字符串是一个由字符组成的串。字符串是只读的。
    \item \textbf{Function}:函数是一段具有名字的代码,它一般用于实现特定的功能。在Berry中函数实际上是一个大类,不过在通常情况下我们不用关心一个``函数''具体是什么子类型。
    \item \textbf{Class}:由一些成员变量和方法函数组成。类用于实现面向对象特性。类是一个抽象的对象并且是只读的。
    \item \textbf{Instance}:一个类实例化之后的对象,实例是一个具体的对象,其成员变量是可读写的。
    \item \textbf{List}:可变数组。可以存储任何类型。
    \item \textbf{Map}:用于存储键值对。值可以是任何类型。
    \item \textbf{Range}:表示一个整数区间。通常用于范围迭代。
\end{itemize}

\section{表达式}

一个表达式(Statement)由一到多个操作数和运算符组成,通过对表达式求值可以得到一个结果。操作数可以是一个字面值、变量、函数调用或者是子表达式。与四则运算类似,运算符的优先级会影响表达式的求值顺序,优先级越高的运算符,其表达式越先求值。

这是一些表达式的例子:
\begin{lstlisting}[language=berry, numbers=none]
a = 1 + 5       # 6
print(a * 2)    # 12
b = type(a)     # 'int'
\end{lstlisting}

\section{语句}

语句分为表达式语句、\texttt{if}语句、\texttt{while}语句、\texttt{for}语句、\texttt{break}语句、\texttt{continue}语句、\texttt{return}语句、\texttt{import}语句、函数定义语句和类定义语句等。

表达式语句可以是一个赋值表达式或者是函数调用表达式。
\begin{lstlisting}[language=berry, numbers=none]
a = 1       # assignment statement
print(a)    # call statement
\end{lstlisting}

除了行注释以外,回车或换行符(``\texttt{\textbackslash r}''和``\texttt{\textbackslash n}'')仅作为空白字符使用,因此一条语句可以是多行的。多行的表达式和单行表达式写法相同,不需要使用专门的续行符号。但是有些时候,解释器可能把本意是两条语句的代码解释成一条语句:
\begin{lstlisting}[language=berry]
a = 1 +
    func()      # wrap line
b = 1 c = 2     # multiple statements
a = c
(b) = 1         # be regarded as a function call
\end{lstlisting}

这些代码中,第1行和第2行是一个多行表达式语句,第3行有两个表达式语句。第4行和第5行本意是两条表达式语句\texttt{a = c}和\texttt{(b) = 1},但是解释器会把它们理解成一条语句:\texttt{a = c(b) = 1},这和我们的本意是相违背的。

这个问题有以下几种解决方案:

\begin{itemize}
    \item 不要在赋值号的左侧使用括号,而且只使用简单的表达式。
    \item 使用符号``\texttt{;}''来显式地分隔语句。
\end{itemize}

\section{块}

一个块(Block)是若干表达式的集合。块可能在控制语句(例如\texttt{if}和\texttt{while}语句)、函数或者方法体中使用。例如:
\begin{lstlisting}[language=berry]
if (isOpen)
    close()
    print('the device was closed')
end
\end{lstlisting}

此处第2行到第3行的语句构成了一个块。事实上,任意一些连续的语句串都可以说是一个块,并且一个语句内部可能嵌套了块,例如第2行到第3行的块就嵌套在\texttt{if}语句中。

\subsection{\texttt{do}语句}

Berry提供一种\texttt{do}语句来封装代码块。

    \chapter{表达式}

\section{简介}

Berry提供一些一元运算符(Unary Operator)和二元运算符(Binary Operator)。例如逻辑与运算符\texttt{\&\&}就是一个二元运算符,而逻辑非运算符\texttt{!}是一个一元运算符。一些运算符可以是一元运算符也可以是二元运算符。例如运算符\texttt{-}在表达式\texttt{-1}中是一元符号,但是在表达式\texttt{1-2}中则是二元的减号。

需要注意操作数是否支持对其使用的运算符。例如,表达式\texttt{'10'+10}是错误的,因为Berry不支持对一个字符串和一个整数做加法。通常,Berry不会执行隐式类型转换,只有整数和实数的加法中存在例外:此时Berry会自动将整数类型提升为实数类型。例如,表达式\texttt{1.1+1}的结果为\texttt{2.1}。

在表达式中,运算符的优先级和结合性决定了表达式的求值顺序。各个运算符的优先级和结合性在表\ref{tab::operator_list}中给出。

优先级指定了不同运算符之间的求值顺序,具有高优先级运算符的表达式会先被求值。例如,对表达式\texttt{1+2*3}的求值过程会先计算\texttt{2*3}的结果,然后计算加法表达式的结果。使用括号可以提升低优先级表达式的求值顺序,例如在表达式\texttt{(1+2)*3}求值中,先计算括号中表达式\texttt{1+2}的结果,然后计算括号外的乘法表达式。

结合性是指运算符两侧操作数的求值顺序,这里的操作数可能是子表达式。例如,在加法表达式\texttt{expr1 + expr2}中,先计算\texttt{expr1}的值再计算\texttt{expr2}的值,这是因为加法运算符是左结合的。

\section{算术运算符}

算术运算符用于实现算术运算,这些运算符和我们平时使用的数学符号相似。Berry提供的算术运算符如表\ref{tab::arthmetic_operator}所示。

\begin{table}[htb]
    \centering
    \setlength{\tabcolsep}{10mm}
    \begin{tabular}{ccc} \Xhline{1pt}
        \makecell[c]{\textbf{运算符}} & \makecell[c]{\textbf{功能}} & \makecell[c]{\textbf{示例}} \\ \Xhline{1pt}
        \texttt{+} & 一元正号 & \texttt{+ expr} \\
        \texttt{-} & 一元负号 & \texttt{- expr} \\
        \texttt{+} & 加号 & \texttt{expr + expr} \\
        \texttt{-} & 减号 & \texttt{expr - expr} \\
        \texttt{*} & 乘号 & \texttt{expr * expr} \\
        \texttt{/} & 除号 & \texttt{expr / expr} \\
        \texttt{\%} & 取余数 & \texttt{expr \% expr} \\
        \Xhline{1pt}
    \end{tabular}
    \caption{算术运算符}
    \label{tab::arthmetic_operator}
\end{table}

对于二元运算符\texttt{+},当操作数为字符串时会执行字符串连接,即把两个字符串连接成一条更长的字符串,其他时候通常用作加法运算符。除了\texttt{\%}运算符以外,其他算数运算符都支持整数和实数操作数,\texttt{\%}运算符只支持整数操作数。\texttt{\%}运算符的作用是计算一个整数整数另一个整数之后的余数,例如\texttt{11\%4}的结果是\texttt{3}。

所有的算数运算符都可以在类中重载,重载后的运算符不一定局限于它们原本的功能设计,而是由程序员自己决定。

正号\texttt{+}实际上没有任何作用。它可能在将来的版本中被移除,因此不建议使用。

\section{关系运算符}

关系运算符用于比较操作数的大小关系。Berry支持的6种关系运算符在表\ref{tab::relop_operator}中给出。

\begin{table}[htb]
    \centering
    \setlength{\tabcolsep}{10mm}
    \begin{tabular}{ccc} \Xhline{1pt}
        \makecell[c]{\textbf{运算符}} & \makecell[c]{\textbf{功能}} & \makecell[c]{\textbf{示例}} \\ \Xhline{1pt}
        \texttt{<} & 小于 & \texttt{expr < expr} \\
        \texttt{<=} & 小于等于 & \texttt{expr <= expr} \\
        \texttt{==} & 小于等于 & \texttt{expr == expr} \\
        \texttt{!=} & 不等于 & \texttt{expr != expr} \\
        \texttt{>=} & 大于等于 & \texttt{expr >= expr} \\
        \texttt{>} & 大于 & \texttt{- expr} \\
        \Xhline{1pt}
    \end{tabular}
    \caption{关系运算符}
    \label{tab::relop_operator}
\end{table}

通过比较操作数的大小关系,关系表达式将产生一个布尔值的结果,当关系满足时,关系表达式的值为\texttt{true},否则为\texttt{false}。关系运算符允许使用以下几种操作数的组合:
\begin{gather*}
    \bm{integer} \quad relop \quad \bm{integer} \\
    \bm{real} \quad relop \quad \bm{real} \\
    \bm{integer} \quad relop \quad \bm{real} \\
    \bm{real} \quad relop \quad \bm{integer} \\
    \bm{string} \quad relop \quad \bm{string}
\end{gather*}

所有的关系运算符都可以在类中重载,因此也可以对实例对象使用关系运算符。

\section{逻辑运算符}

逻辑运算符分为逻辑与、逻辑或和逻辑非3种。如表\ref{tab::logic_operator}所示。

\begin{table}[htb]
    \centering
    \setlength{\tabcolsep}{10mm}
    \begin{tabular}{ccc} \Xhline{1pt}
        \makecell[c]{\textbf{运算符}} & \makecell[c]{\textbf{功能}} & \makecell[c]{\textbf{示例}} \\ \Xhline{1pt}
        \texttt{\&\&} & 逻辑与 & \texttt{expr \&\& expr} \\
        \texttt{||} & 逻辑或 & \texttt{expr || expr} \\
        \texttt{!} & 逻辑非 & \texttt{! expr} \\
        \Xhline{1pt}
    \end{tabular}
    \caption{逻辑运算符}
    \label{tab::logic_operator}
\end{table}

对于逻辑与运算符,当两个操作数都为\texttt{true}时,逻辑表达式的值为\texttt{true},否则为\texttt{false}。

对于逻辑或运算符,当两个操作数都为\texttt{false}时,逻辑表达式的值为\texttt{false},否则为\texttt{true}。

逻辑非运算符会翻转操作数的逻辑状态。当操作数为\texttt{true}时,逻辑表达式的值为\texttt{false},否则值为\texttt{true}。

逻辑运算符要求操作数是布尔类型,否则将尝试使用下列规则进行转换:

\begin{itemize}
    \item Nil:转换为\texttt{false}。
    \item Integer:值为\texttt{0}时转换为\texttt{false},否则转换为\texttt{true}。
    \item Real:值为\texttt{0.0}时转换为\texttt{false},否则转换为\texttt{true}。
    \item Instance:存在方法\texttt{tobool()}时将使用该方法的返回值,否则转换为\texttt{true}。
    \item 其他:转换为\texttt{true}。
\end{itemize}

\section{赋值运算符}

赋值运算符\texttt{=}仅出现在赋值表达式中,其左操作数必须是可写对象。赋值表达式不返回值,因此不能使用连续的赋值运算,类似\texttt{a=b=c}这样的表达式是错误的。

\section{域运算符和下标运算符}

域运算符\texttt{.}用于访问对象的一个属性或者成员,你可以对模块和实例这两种类型来使用域运算符:
\begin{lstlisting}[language=berry, numbers=none]
l = list[]
l.append('item 0')
s = l.item(0)       # 'item 0'
\end{lstlisting}

下标运算符\texttt{[]}用于访问对象的元素,例如
\begin{lstlisting}[language=berry, numbers=none]
l[2] = 10   # read by index
n = l[2]    # write by index
\end{lstlisting}

    \chapter{语句}

Berry是一种命令式编程语言,这种范式认为程序是一步一步被执行的。通常情况下,Berry语句是顺序执行的,这种程序结构称为顺序结构。尽管顺序结构非常基础,但是实际的程序中通常还需要用到分支结构和循环结构,Berry提供几种控制语句来实现这种复杂的流程结构,例如条件语句和迭代语句。

除了行注释以外,回车符或换行符(``\texttt{\textbackslash r}''和``\texttt{\textbackslash n}'')仅作为空白字符使用,因此语句可以跨行书写。此外还可以将多个语句写在同一行。

可以在语句的末尾加一个分号来表示语句结束,不过解释器通常能够自动切分语句而无需使用分号。对于语句分割会出现歧义的代码可以使用分号来告诉解释器该怎样解析代码,不过,更好的建议是不要去写有歧义的代码。

\section{简单语句}

\subsection{表达式语句}

表达式语句主要是赋值表达式或者函数调用表达式构成的语句。其他表达式也可以构成语句,但是没有意义,例如表达式 \texttt{1+2} 单独写就是一条语句,但它没有任何作用。下面的例程给出了表达式语句和函数语句的例子:
\begin{lstlisting}[language=berry, numbers=none]
a = 1       # Assignment statement
print(a)    # Call statement
\end{lstlisting}
第2行是一个简单的赋值语句,该语句将字面值 \texttt{i} 赋值给变量 \texttt{a}。第2行语句是函数调用语句,该语句通过调用 \texttt{print} 函数打印变量 \texttt{a} 的值。

跨行表达式和单行表达式写法相同,不需要使用特殊的续行符号。例如:
\begin{lstlisting}[language=berry, numbers=none]
a = 1 +
    func()      # Wrap line
\end{lstlisting}
也可以将多个表达式语句写在一行,各种类型的语句都可以写在一行。这个例子就把两个表达式语句写在同一行:
\begin{lstlisting}[language=berry, numbers=none]
b = 1 c = 2     # Multiple statements
\end{lstlisting}

有些时候程序员希望写两条语句,但是解释器可能会错误的认为是一条语句。这个问题是由于语法解析的过程中存在歧义而产生的。以这段代码为例:
\begin{lstlisting}[language=berry, numbers=none]
a = c
(b) = 1         # Be regarded as a function call
\end{lstlisting}
假设第4行和第5行本意是两条表达式语句:\texttt{a = c} 和 \texttt{(b) = 1},然而解释器会把它们理解成一条语句:\texttt{a = c(b) = 1}。这个问题的成因在于解释器错误地将 \texttt{c} 和 \texttt{(b)} 解析成函数调用。为了避免歧义,我们可以在语句末尾加上分号来明确地分隔语句:
\begin{lstlisting}[language=berry, numbers=none]
a = c; (b) = 1;
\end{lstlisting}
更好的办法是不要在赋值号左侧使用括号,很明显,在这里使用括号显得毫无理由。通常情况下,赋值运算符左边不应该出现复杂的表达式,而只有由变量名、域运算表达式、下标运算表达式等构成的简单表达式:
\begin{lstlisting}[language=berry, numbers=none]
a = c  b = 1
\end{lstlisting}
只在赋值号左边使用简单表达式不会引起语句断句方面的歧义。因此大部分场合下无需使用分号来分割表达式,我们也不推荐这种写法。

\subsection{块} \label{section::block}

一个\textbf{块}(block)是若干语句的集合。一个块就是一个作用域,因此在块中定义的变量只能在块及其子块内部访问。有很多地方使用块,例如 \texttt{if} 语句、 \texttt{while} 语句、函数声明等。这些语句都会通过一对关键字来包含一个块。例如在 \texttt{if} 语句中使用的块:
\begin{lstlisting}[language=berry]
if isOpen
    close()
    print('the device was closed')
end
\end{lstlisting}
第2行到第3行的语句构成了一个块,这个块夹在 \texttt{if} 和 \texttt{end} 这一对关键字中间( \texttt{if} 语句的条件表达式不在块内)。块也可以不包含任何语句,这样就构成一个空的块,也可以说是包含一条空语句的块。宽泛的说,可以讲任意几条连续的语句称为一个块,但我们更倾向于尽量扩大块的范围,这可以保证块的区域和作用域的范围一致。在上面的例子中,我们倾向于认为第2至3行是一整个块,这是 \texttt{if} 关键字到 \texttt{end} 关键字之间最大的范围。

\subsubsection{\texttt{do}语句}

有时候我们仅仅想开辟一个新的作用域,但是又不想使用任何控制语句,这种情况下可以使用 \texttt{do} 语句来封装块。\texttt{do} 语句没有任何控制功能。\texttt{do} 语句的形式为
\begin{algorithm}
    \texttt{do}\\
    \qquad $\bm{block}$ \\
    \texttt{end}
\end{algorithm}\vspace{-0.6em}\\
其中$\bm{block}$就是我们需要的块。该语句使用一对 \texttt{do} 和 \texttt{end} 关键字来包含块。\texttt{do} 语句没有控制功能,也不会生成任何运行时的指令。

\section{条件语句}

Berry提供 \texttt{if} 语句来实现条件控制执行的功能,这类程序结构一般被称为分支结构。\texttt{if} 语句会根据条件表达式的真(\texttt{true})或假(\texttt{false})来决定执行的分支。在一些语言中,实现条件控制还有其他的选择,例如C、C++等语言提供 \texttt{switch} 语句,不过为了简化设计,Berry没有支持 \texttt{switch} 语句。

\subsection{\texttt{if} 语句}

\textbf{\texttt{if} 语句}用于实现分支结构,它根据某种判断条件的真假来选择程序运行的分支,该语句还可以包含 \texttt{else} 分支或者 \texttt{elif} 分支。不包含分支的简单 \texttt{if} 语句形式为
\begin{algorithm}
    \texttt{if} $\bm{condition}$ \\
    \qquad $\bm{block}$ \\
    \texttt{end}
\end{algorithm}\vspace{-0.6em}\\
$\bm{condition}$是条件表达式,当$\bm{condition}$的值为 \texttt{true} 时,将会执行第二行的$\bm{block}$,否则会跳过该$\bm{block}$并执行 \texttt{end} 后面的语句。在$\bm{block}$被执行的情况下,块中的最后一条语句执行完以后会离开 \texttt{if} 语句并开始执行 \texttt{end} 后面的语句。

下面通过一个例子来说明 \texttt{if} 语句的用法:
\begin{lstlisting}[language=berry, numbers=none]
if 8 % 2 == 0
    print('this number is even')
end
\end{lstlisting}
这段代码用于判断数字 \texttt{8} 是否是偶数,如果是则会输出 \texttt{this number is even}。这个例子虽然十分简单,却足够说明 \texttt{if} 语句的基本使用方法。

如果希望条件满足和不满足时都有对应的分支可供执行则要使用具有 \texttt{else} 分支的 \texttt{if} 语句。\texttt{if else} 语句的形式为
\begin{algorithm}
    \texttt{if} $\bm{condition}$ \\
        \qquad $\bm{block}$ \\
    \texttt{else} \\
        \qquad $\bm{block}$ \\
    \texttt{end}
\end{algorithm}\vspace{-0.6em}
与简单的 \texttt{if} 语句不同的是,\texttt{if else} 语句在$\bm{condition}$的值为 \texttt{false} 时将会执行 \texttt{else} 分支下的$\bm{block}$。无论执行哪个分支下面的$\bm{block}$,块中的最后一条语句执行完以后都会跳出 \texttt{if else} 语句,也就是执行 \texttt{end} 后面的语句。也就是说,无论$\bm{condition}$的值是 \texttt{true} 还是 \texttt{false},都会有一个$\bm{block}$被执行。

继续用判断奇偶数作为例子,这次把需求改为根据输入数字的奇偶性输出对应的信息。实现这项需求的代码是:
\begin{lstlisting}[language=berry, numbers=none]
if x % 2 == 0
    print('this number is even')
else
    print('this number is odd')
end
\end{lstlisting}
运行这段代码之前要先给变量 \texttt{x} 赋一个整数值,这个值就是我们要检测奇偶的数。假如 \texttt{x} 是一个偶数,程序将会输出 \texttt{this number is even},否则输出 \texttt{this number is odd}。

有时候我们需要嵌套使用 \texttt{if} 语句,一种情是要在 \texttt{else} 分支下嵌套一个 \texttt{if} 语句,这是一种很常见的需求,因为很多时候要连续判断多个条件。对于这种需求,使用 \texttt{if else} 语句的写法是:
\begin{lstlisting}[language=berry, numbers=none]
if expr
    block
else
    if expr
        block
    end
end
\end{lstlisting}
很明显,这种写法会增加代码的缩进层次,并且结尾处要使用多个 \texttt{end},比较繁琐。作为改进,Berry提供了 \texttt{elif} 分支来优化上面的写法。使用 \texttt{elif} 分支与上述代码等价,其形式为
\begin{algorithm}
    \texttt{if} $\bm{condition}$ \\
        \qquad $\bm{block}$ \\
    \texttt{elif} $\bm{condition}$ \\
        \qquad $\bm{block}$ \\
    \texttt{else} \\
    \qquad $\bm{block}$ \\
    \texttt{end}
\end{algorithm}\vspace{-0.6em}

\texttt{elif} 分支必须用在 \texttt{if} 分支之后以及\text{else}分支之前,并且 \texttt{elif} 分支可以连续使用多次。如果 \texttt{elif} 分支对应的$\bm{condition}$满足则会执行分支下的$\bm{block}$。\texttt{elif} 分支适合于需要多种条件依次判断的场合。

我们用一段判断正数、负数和0的代码来演示 \texttt{elif} 分支:
\begin{lstlisting}[language=berry, numbers=none]
if x > 0
    print('positive')
elif x == 0
    print('zero')
else
    print('negative')
end
\end{lstlisting}
这里同样要先为变量 \texttt{x} 赋值。这段代码非常简单,不再解释。

有一些语言存在一种称为悬垂``\texttt{else}''的问题,该问题是指当一个 \texttt{if} 语句嵌套在另一个 \texttt{if} 语句内部的时候, \texttt{else} 分支到底归属于那条 \texttt{if} 语句的问题。使用C/C++时就要考虑悬垂 \texttt{else} 的问题,C/C++程序员为了避免在 \texttt{if else} 的问题上出现歧义往往会使用花括号来把分支做成一个块。在Berry中,\texttt{if} 语句的分支必然是一个块,这也就决定了Berry没有悬垂 \texttt{else} 的问题。

\section{迭代语句}

迭代语句又叫循环语句,用于重复执行某种操作直到终止条件满足。Berry提供 \texttt{while} 语句和 \texttt{for} 两种迭代语句。很多语言也是提供这两种语句用于迭代。Berry的 \texttt{while} 语句和C/C++中的 \texttt{while} 语句类似,但是Berry的 \texttt{for} 语句仅用于遍历容器中的元素,类似于一些语言提供的 \texttt{foreach} 语句,以及C++ 11引入的新 \texttt{for} 语句风格。而C风格的 \texttt{for} 语句没有支持。

\subsection{\texttt{while} 语句}

\textbf{\texttt{while} 语句}是一种基本的迭代语句,\texttt{while} 语句会使用一个判定条件,当条件为真时就重复执行循环体,否则结束循环。该语句的模式为
\begin{algorithm}
    \texttt{while} $\bm{condition}$ \\
        \qquad $\bm{block}$ \\
    \texttt{end}
\end{algorithm}\vspace{-0.6em}\\
程序运行到 \texttt{while} 语句时会检测$\bm{condition}$表达式的真假,如果为真则执行循环体$\bm{block}$,否则结束循环。执行完$\bm{block}$中的最后一条语句后,程序会跳转到 \texttt{while} 语句的开始位置并开始下一轮的检测。如果$\bm{condition}$表达式在第一次求值时为假会导致循环体$\bm{block}$一次都不会执行(和 \texttt{if} 语句的$\bm{condition}$表达式为假一样)。

通常而言,$\bm{condition}$表达式的值应该可以在循环过程中改变,而不应该是一个常量或者在循环外修改的变量,这样将会导致循环不会执行或者无法终止。永不终止的循环称为死循环,通常我们通常期望循环执行指定的次数然后终止。例如在使用 \texttt{while} 循环来访问数组中的所有元素时,我们希望循环执行的次数为数组的长度,例如:
\begin{lstlisting}[language=berry, numbers=none]
i = 0
l = ['a', 'b', 'c']
while i < l.size()
    print(l[i])
    i = i + 1
end
\end{lstlisting}
这个循环从数组 \texttt{l} 中获取元素并打印,我们使用一个变量 \texttt{i} 来做循环计数器以及数组下标。我们让 \texttt{i} 的值达到数组 \texttt{l} 的长度时结束循环。在循环体的最后一行,我们让 \texttt{i} 的值加 \texttt{1},这样可以保证下次循环访问数组的下一个元素,并使 \texttt{while} 循环在循环次数达到数组长度时结束。

\subsection{\texttt{for} 语句}

Berry的\textbf{\texttt{for} 语句}用于遍历容器中的元素,其形式为
\begin{algorithm}
    \texttt{for } $\bm{varaible}$ \texttt{:} $\bm{expression}$ \\
        \qquad $\bm{block}$ \\
    \texttt{end}
\end{algorithm}\vspace{-0.6em}

$\bm{expression}$表达式的值必须是一个可迭代的容器或者函数,例如 \texttt{range} 类。\texttt{for} 语句从容器中获取一个迭代器,通过对迭代器的调用实现每次循环得到容器中的一个元素。

$\bm{varaible}$称为迭代变量,迭代变量总是在 \texttt{for} 语句中定义。因此$\bm{varaible}$必须是一个变量名而不能是表达式。每次循环时从迭代器中获取的容器元素将会赋值给迭代变量,该过程发生在$\bm{block}$的第一条语句之前。

\texttt{for} 语句会检查迭代器中是否还有未访问的元素可供迭代,如果有则开始下一轮的迭代,否则会结束 \texttt{for} 语句并执行 \texttt{end} 后面的语句。目前,Berry只提供只读迭代器,也就是说无法通过 \texttt{for} 语句中的迭代变量来修改容器中的元素。

迭代变量$\bm{varaible}$的作用域仅限于循环体$\bm{block}$中,该变量不会和作用域外的同名变量有任何关系。为了说明这一点,我们通过一个例子来说明,在这个例子里,我们使用 \texttt{for} 语句来访问 \texttt{rang} 实例中的所有元素并打印出来。当然,我们还通过这个例子来演示循环变量的作用域。
\begin{lstlisting}[language=berry]
i = "Hi, I'm fine." # Outer variable
for i : 0 .. 2
    print(i)        # Iteration variable
end
print(i)
\end{lstlisting}

在这个例子中,相对于第2行定义的的迭代变量 \texttt{i} 而言,第1行定义的变量 \texttt{i} 为外部变量。运行这个例子将得到下面的结果
\begin{lstlisting}[numbers=none]
0
1
2
Hi, I'm fine
\end{lstlisting}
可以看出,迭代变量 \texttt{i} 和外部变量 \texttt{i} 是两个不同的变量,它们只是名字相同,但是作用域不同。

\subsubsection{\texttt{for} 语句原理}

与 \texttt{while} 这种传统的迭代语句不同,\texttt{for} 语句使用迭代器来遍历容器。如果需要对自定义的类使用 \texttt{for} 语句进行遍历,就需要了解它的实现机制。在使用 \texttt{for} 语句的时候,解释器隐藏了很多实现细节,实际上,对于这样的代码:
\begin{lstlisting}[language=berry]
for i : 0 .. 2
    print(i)
end
\end{lstlisting}
会被解释器翻译为以下等价代码:
\begin{lstlisting}[language=berry]
var it = __iterator__(0 .. 2)
try
    while true
        var i = it()
        print(i)
    end
except 'stop_iteration'
    # do nothing
end
\end{lstlisting}

从某种程度上讲,\texttt{for} 语句只是一个语法糖,它实质上只是一段复杂代码的简便写法。在这段等效代码中使用了一个中间变量 \texttt{it},该变量的值是一个迭代器,在这个例子里,它是 \texttt{range} 容器 \texttt{0..2} 的迭代器。在处理 \texttt{for} 语句时,解释器会将迭代器中间变量隐藏起来,因此无法在代码中访问它。

% <UNUSED>
\if{false}
函数 \texttt{\_\_iterator\_\_} 的参数是一个容器,该函数会返回参数的迭代器。这个函数通过调用参数的 \text{iter} 方法来得到迭代器。因此,\texttt{iter} 方法的返回值要是实例(\texttt{instance})类型,这个实例要有 \texttt{next} 方法和 \texttt{hasnext} 方法。

函数 \texttt{\_\_hasnext\_\_} 的参数是一个迭代器,它通过调用迭代器的 \texttt{hasnext} 方法来检查迭代器是否有下一个元素。\texttt{hasnext} 方法的返回值是 \texttt{boolean} 类型。函数 \texttt{\_\_next\_\_} 的参数也是迭代器,它通过调用迭代器的 \texttt{next} 方法获取迭代器中的下一个元素。

目前为止,\texttt{\_\_iterator\_\_}、\texttt{\_\_hasnext\_\_} 和 \texttt{\_\_next\_\_} 函数都只是简单地调用容器或者迭代器的一些方法然后返回这些方法的返回值。因此,\texttt{for} 语句的等效写法还可以简化成这种形式:
\begin{lstlisting}[language=berry]
do
    var it = (0 .. 2).iter()
    while (it.hasnext())
        var i = it.next()
        print(i)
    end
end
\end{lstlisting}
这段代码就比较易读了。从效代码中可以看出:迭代器变量 \texttt{it} 的作用域为整个 \texttt{for} 语句,但是在 \texttt{for} 语句外不可见,而迭代变量 \texttt{i} 的作用域在循环体内,因此每次迭代都会定义新的迭代变量。
\fi
% !<UNUSED>

\section{跳转语句}

Berry提供的的跳转语句用于在循环过程中实现程序流程的跳转。跳转语句分为 \texttt{break} 语句和 \texttt{continue} 语句,这两种语句必须在迭代语句内部使用,并且只能在函数内部实现跳转。有些语言提供 \texttt{goto} 语句来实现函数内部的任意跳转,这种语句Berry不提供,不过 \texttt{goto} 语句的效果完全可以用条件语句和迭代语句来代替。

\subsection{\texttt{break} 语句}

\texttt{break} 用来终止迭代语句并跳出。执行 \texttt{break} 语句之后将立即终止最近一层迭代语句并从该迭代语句后的第一条语句位置继续执行。为了说明 \texttt{break} 语句的执行流程,我们使用一个例子来演示:
\begin{lstlisting}[language=berry]
while true
    print('before break')
    break
    print('after break')
end
print('out of the loop')
\end{lstlisting}
在这段代码中, \texttt{break} 语句处于一个 \texttt{while} 循环中,在 \texttt{break} 语句前后和 \texttt{while} 语句后面我们各放置了一条打印语句来测试程程序的执行流程。这段代码的运行结果为:
\begin{lstlisting}[numbers=none]
before break
out of the loop
\end{lstlisting}
这说明 \texttt{while} 语句在第3行的 \texttt{break} 语句位置结束了循环并使程序从第6行继续执行。

\subsection{\texttt{continue} 语句}

 \texttt{continue} 语句也在迭代语句内部使用,它的作用是结束一次迭代并立即开始下一轮。因此,执行 \texttt{continue} 语句之后,最近一层的迭代语句中的剩余代码将不再执行,而是开始新一轮迭代。这里我们使用一个 \texttt{for} 语句来演示 \texttt{continue} 语句的功能:
\begin{lstlisting}[language=berry]
for i : 0 .. 5
    if i >= 2
        continue
    end
    print('i =', i)
end
print('out of the loop')
\end{lstlisting}
这里, \texttt{for} 语句将会迭代6次,当迭代变量 \texttt{i} 大于等于 \texttt{2} 之后将执行第3行的 \texttt{continue} 语句,此后第5行的打印语句将不会执行。也就是说第5行只会在前两轮迭代中执行(此时\texttt{i<2} )。该例程的运行结果为:
\begin{lstlisting}[numbers=none]
i = 0
i = 1
out of the loop
\end{lstlisting}
可以看出变量 \texttt{i} 的值只打印了2次,符合预期。读者可以尝试在 \texttt{continue} 语句之前打印变量 \texttt{i} 的值,你会发现 \texttt{for} 语句确实迭代了6次,说明 \texttt{continue} 语句并不会终止迭代。

\section{\texttt{import} 语句}

Berry有一些预先定义的模块(module),例如用于数学计算的 \texttt{math} 模块。这些模块不能直接使用,而是要用 \texttt{import} 语句来导入。导入一个模块的方法有两种:
\begin{algorithm}
    \texttt{import} $\bm{name}$ \\
    \texttt{import} $\bm{name}$ \texttt{as} $\bm{varaible}$
\end{algorithm}\vspace{-0.6em}\\
$\bm{name}$为需要导入的模块名字,使用第1种写法来导入模块时,在直接使用模块名即可调用被导入的模块。而第二种写法是导入了一个名为$\bm{name}$的模块并将调用时的模块名修改为$\bm{varaible}$。例如一个名为 \texttt{math} 的模块,我们使用第一种方法来导入并使用:
\begin{lstlisting}[language=berry, numbers=none]
import math
math.sin(0)
\end{lstlisting}
这里直接使用 \texttt{math} 就可以调用模块了。如果一个模块的名字比较长,不便于书写则可以使用 \texttt{import as} 语句,这里假设一个名为 \texttt{hardware} 的模块。我们要调用该模块的 \texttt{setled} 函数,可以把模块 \texttt{hardware} 导入到名为 \texttt{hw} 的变量中并使用:
\begin{lstlisting}[language=berry, numbers=none]
import hardware as hw
hw.setled(true)
\end{lstlisting}

\section{异常处理}

\textbf{异常处理}机制允许程序对运行时出现的异常进行捕获并进行处理。Berry 支持异常捕获机制,它允许将异常的捕获和处理过程分离开。即一部分程序用于检测并搜集异常,另一部分程序用于处理异常。

首先,出现问题的程序需要先抛出异常,当这些程序位于一个异常处理块中时,将由特定的程序去捕获并处理该异常。

\subsection{抛出异常}

使用 \texttt{raise} 语句引发一个异常。\texttt{raise} 语句会传递一个值用于表示异常的类型,以便于特定的异常处理程序识别。以下是 \texttt{raise} 语句的使用方法:
\begin{algorithm}
    \texttt{raise }$\bm{exception}$ \\
    \texttt{raise }$\bm{exception}$\texttt{, }$\bm{message}$
\end{algorithm}\vspace{-0.6em}\\
表达式 $\bm{exception}$ 的值就是抛出的\textbf{异常值};可选的 $\bm{message}$ 表达式通常是一条说明异常信息的字符串,该表达式被称为\textbf{异常参数}。Berry 允许将任何值作为异常值,例如可以将字符串作为异常值:
\begin{lstlisting}[language=berry, numbers=none]
raise 'my_error', 'an example of raise'
\end{lstlisting}

程序执行到 \texttt{raise} 语句后就不会继续执行它后面的语句,而是会跳转到最近的异常处理块。如果最近的异常处理块在其他函数中,则沿着调用链的函数会提早退出。如果没有异常处理块则会发生\textbf{异常退出},此时解释器会打印异常报错信息和报错位置的调用栈。

当 \texttt{raise} 语句处于 \texttt{try} 语句块中时,异常将被后者捕获。被捕获的异常将由与 \texttt{try} 块关联的 \texttt{except} 块去处理。如果抛出的异常可以被 \texttt{except} 块处理,那么执行完这个块后会从最后一个 \texttt{except} 块后的语句处继续执行。如果所有 \texttt{except} 语句都不能处理这个异常,那么该异常会重新抛出,直到能够被处理或者异常退出。

\subsubsection{异常值}

在 Berry 中可以把任意值作为异常值,不过我们通常会使用简短的字符串。Berry 内部也可能抛出一些异常,我们把这些异常称为\textbf{标准异常},所有的标准异常值都是字符串类型,表 \ref{tab::stdexpect_list} 列出了所有的标准异常。
\begin{table}[htb]
    \centering
    \setlength{\tabcolsep}{3mm}
    \begin{tabular}{cll} \toprule
        \textbf{异常值} & \textbf{说明} & \textbf{参数说明} \\ \midrule
        \texttt{assert\_failed} & 断言失败 & 具体异常信息 \\
        \texttt{index\_error} & 下标错误(通常是越界) & 具体的异常信息 \\
        \texttt{io\_error} & IO 功能异常 & 具体异常信息 \\
        \texttt{key\_error} & 键错误 & 具体异常信息 \\
        \texttt{runtime\_error} & VM 运行时异常 & 具体异常信息 \\
        \texttt{stop\_iteration} & 迭代器结束 & 无 \\
        \texttt{syntax\_error} & 语法解析错误 & 编译器给出的具体报错信息 \\
        \texttt{unrealized\_error} & 未实现的功能 & 具体的异常信息 \\
        \texttt{type\_error} & 类型错误 & 具体异常信息 \\
        \bottomrule
    \end{tabular}
    \caption{标准异常列表}
    \label{tab::stdexpect_list}
\end{table}

\subsection{捕获异常}

使用 \texttt{excpet} 语句可以捕获异常,它必须和 \texttt{try} 语句配对使用,即一个 \texttt{try} 语句块后面必须跟随一至多个 \texttt{except} 语句块。\texttt{try-except} 语句的基本形式为
\begin{algorithm}
    \texttt{try} \\
        \qquad $\bm{block}$ \\
    \texttt{excpet} $\bm{...}$ \\
        \qquad $\bm{block}$ \\
    \texttt{end}
\end{algorithm}\vspace{-0.6em}\\
其中 \texttt{except} 分支可以具有以下几种形式
\begin{algorithm}
    \texttt{excpet ..} \\
    \texttt{excpet }$\bm{exceptions}$ \\
    \texttt{excpet }$\bm{exceptions}$\texttt{ as }$\bm{variable}$ \\
    \texttt{excpet }$\bm{exceptions}$\texttt{ as }$\bm{variable}$\texttt{, }$\bm{message}$ \\ 
    \texttt{excpet .. as }$\bm{variable}$ \\
    \texttt{excpet .. as }$\bm{variable}$\texttt{, }$\bm{message}$ \\
\end{algorithm}\vspace{-0.6em}\\
最基础的 \texttt{except} 语句不使用参数,此 \texttt{except} 分支将会捕获所有的异常;\textbf{捕获异常列表} $\bm{exceptions}$ 为对应 \texttt{except} 分支可以匹配的异常值列表,列表中多个值之间使用逗号分隔;$\bm{variable}$ 为\textbf{异常变量},如果该分支捕获了异常则会将异常值绑定到该变量;$\bm{message}$ 为\textbf{异常参数变量},如果该分支捕获了异常则会将异常参数值绑定到该变量。

当 \texttt{try} 语句块捕获到一条异常时,解释器会逐个检测 \texttt{except} 分支。如果某分支的捕获列表存在该异常值则会调用该分支下的代码块以处理这个异常,代码块执行完后会退出整个 \texttt{try-except} 语句。如果所有的 \texttt{except} 分支都不匹配,该异常将被重新抛出并由更外层的异常处理程序捕获和处理。

    \chapter{函数}

\textbf{函数}(function)是一种可以被外部代码调用的“子程序”,作为程序的一部分,函数本身也是一段代码。函数可以具有0到多个参数,并且会返回一个结果,这个结果称为函数的\textbf{返回值}。

在Berry中,函数是\textbf{第一类值}(first class value)。因此,除了调用函数以外,你还可以把函数作为值传递,例如将函数绑定到变量,将函数作为返回值等等。

\section{基本信息}

函数的使用包括函数的定义和调用两个部分。函数定义语句使用 \texttt{def} 关键字作为开头,函数定义是将函数体的代码打包并命名的过程,这个过程仅仅生成函数结构而不会执行函数。执行函数须使用\textbf{调用运算符}(call operator),该运算符是一对圆括号。调用运算符作用于一个结果是函数类型的表达式,传给函数的参数写在圆括号之内,多个参数之间使用逗号隔开。调用表达式的结果就是函数的返回值。

\subsection{函数定义}

\subsubsection{具名函数}

\textbf{具名函数}(named function)是定义时赋予了名字的函数,其定义语句由以下几部分构成:\texttt{def} 关键字、函数名、由0到多个参数(parameter)组成的列表以及函数体(function body),参数列表中的多个参数以逗号分隔,所有参数写在一对圆括号中。我们把函数定义时的参数称为\textbf{形参},而调用函数时的参数称为\textbf{实参}。函数定义的一般形式为:
\begin{algorithm}
    \texttt{def} $\bm{name}$ \texttt{(} $\bm{arguments}$ \texttt{)} \\
        \qquad $\bm{block}$ \\
    \texttt{end}
\end{algorithm}\vspace{-0.6em}\\
函数名$\bm{name}$是一个标识符;$\bm{arguments}$为形参列表;$\bm{block}$为函数体,如果函数体为空语句则函数被称为“空函数”。函数返回值语句包含在函数体中,如果$\bm{block}$中没有返回语句,函数默认会返回 \texttt{nil} 返回值。函数名实际上是绑定函数对象的变量名称,如果当前的作用域已经存在这个名称,定义函数相当于把函数对象绑定到这个变量。

下面的例子定义了一个名为 \texttt{add} 的函数,该函数的功能是求两个数的和并返回。
\begin{lstlisting}[language=berry, numbers=none]
def add(a, b)
    return a + b
end
\end{lstlisting}
\texttt{add} 函数具有两个参数 \texttt{a} 和 \texttt{b},两个被加数便通过这些参数传入函数进行计算。\texttt{return} 语句会返回计算的结果。

作为类属性的函数称为方法,这部分内容会在面向对象章节中说明。

\subsubsection{匿名函数}

与具名函数不同,\textbf{匿名函数}(anonymous function)没有名字,其定义表达式的形式为:
\begin{algorithm}
    \texttt{def} \texttt{(} $\bm{arguments}$ \texttt{)} \\
        \qquad $\bm{block}$ \\
    \texttt{end}
\end{algorithm}\vspace{-0.6em}\\
可以看出,与具名函数相比,匿名函数的定义中没有函数名$\bm{name}$。匿名函数的定义实质上是一个表达式,该表达式称为\textbf{函数字面值}。为了使用匿名函数,我们可以将函数字面值绑定到一个变量:
\begin{lstlisting}[language=berry, numbers=none]
add = def (a, b)
    return a + b
end
\end{lstlisting}
这段代码的功能和上一小节中 \texttt{add} 函数的功能完全相同。使用匿名函数可以方便地以字面值的形式传递函数值。与其他类型的字面量一样,函数字面量也是表达式的最小单元,因此从 \texttt{def} 关键字之间 \texttt{end} 之间是一个不可分割的整体。

\subsection{调用函数}

以 \texttt{add} 函数为例,调用该函数需要提供两个数值,通过调用函数可以得到两数之和:
\begin{lstlisting}[language=berry, numbers=none]
res = add(5, 3)
print(res)      # 8
\end{lstlisting}
我们把被调用的函数(例中的 \texttt{add} 函数)称为\textbf{被调函数},而调用被调函数的函数(例中为 \texttt{main} 函数)称为\textbf{主调函数}。函数调用的过程为:首先解释器会(隐式地)使用实参列表初始化被调函数的形参列表,同时暂停主调函数并保存其状态,接下来为被调函数创建环境并执行被调函数。

函数会在遇到 \texttt{return} 语句时结束执行并将返回值传递给主调函数。解释器会在被调函数返回后销毁被调函数的环境,然后恢复主调函数的环境并继续执行主调函数。函数的返回值也是函数调用表达式的结果。

下面的例子定义了一个函数 \texttt{square} 并把这个函数绑定到变量 \texttt{f},然后通过变量 \texttt{f} 来调用 \texttt{square} 函数函数。这种用法类似于C语言的函数指针。
\begin{lstlisting}[language=berry, numbers=none]
def square(n)
    return n * n
end
f = square
print(f(5))     # 25
\end{lstlisting}
需要注意的是,函数对象只是绑定到这些变量(参考\ref{section::assign_operator}节)并且不可修改,因此对函数名对应的变量重新赋值并不会使这个函数丢失:
\begin{lstlisting}[language=berry, numbers=none]
f = square
square = nil
print(f(5))     # 25
\end{lstlisting}
可以看到对 \texttt{square} 重新赋值后函数依然能正常调用。只有函数对象不再与任何变量绑定之后才会丢失,这类函数对象占用的资源将会被系统回收。

\subsubsection{前向调用}

函数的调用必须位于函数变量的作用域内,因此在函数定义之前通常不能调用。为了解决这个问题可以使用这种方法来折衷:
\begin{lstlisting}[language=berry]
var func1
def func2(x)
    return func1(x)
end
def func1(x)
    return x * x
end
print(func2(4))     # 16
\end{lstlisting}
在这个示例中, \texttt{func2} 调用了 \texttt{func1},而函数 \texttt{func1} 的定义却在 \texttt{func2} 之后。执行这段代码后,程序将会输出正确结果 \texttt{16}。这个例程利用了函数定义时不会被调用的机制,在定义 \texttt{func2} 之前先定义变量 \texttt{func1},这样可以保证编译时不会找不到符号 \texttt{func1}。然后我们在 \texttt{func2} 之后定义函数 \texttt{func1},这样会使用该函数来覆盖变量 \texttt{func1} 的值。最后一行 \texttt{print(func2(4))} 中调用函数 \texttt{func2} 时,变量 \texttt{func1} 已经是我们需要的函数,因此会输出正确结果。

\subsubsection{递归调用}

\textbf{递归函数}是指会直接或者间接调用自身的函数。递归是指一种将问题划分为同类子问题然后解决的策略。以阶乘为例,阶乘的递归定义为$0!=1, n!=n\cdot(n-1)!$,我们可以根据定义写出用于计算阶乘的递归函数:
\begin{lstlisting}[language=berry]
def fact(n)
    if (n == 0)
        return 1
    end
    return n * fact(n - 1)
end
\end{lstlisting}
以$5$的阶乘为例,手工计算5的阶乘的过程为:
\begin{equation*}
5! = 5 \times 4 \times 3 \times 2 \times 1 = 120
\end{equation*}
调用 \texttt{fact} 函数得到结果也是$120$:
\begin{lstlisting}[language=berry, numbers=none]
print(fact(5))  # 120
\end{lstlisting}

为了保证递归调用的深度有限(递归层次过深会耗尽栈空间),递归函数必须有一个结束条件。\texttt{fact} 函数定义中第2行的 \texttt{if} 语句用于结束条件的检测,当计算到 \texttt{n} 为 \texttt{0} 时递归过程结束。上述阶乘公式不适用于非整数参数,执行类似 \texttt{fact(5.1)} 的表达式将会因无法结束递归而发生栈溢出错误。

还有一种情况是 \texttt{间接递归},也就是函数不是由它自己调用而是由它调用的另一个函数(直接或间接)调用。间接递归时通常需要使用函数前向调用的技巧,以计算奇数和偶数的函数 \texttt{is\_odd} 和 \texttt{is\_even} 函数为例:
\begin{lstlisting}[language=berry]
var is_odd
def is_even(n)
    if (n == 0)
        return true
    end
    return is_odd(n - 1) 
end
def is_odd(n)
    if (n == 0)
        return false
    end
    return is_even(n - 1) 
end
\end{lstlisting}
这两个函数互相调用了对方。为了保证第6行调用 \texttt{is\_odd} 函数时作用域中有这个名称,在第1行定义了变量 \texttt{is\_odd}。

\subsubsection{匿名函数调用}

如果一个匿名函数只会被调用一次,最简单的办法就是在定义的时候调用,例如:
\begin{lstlisting}[language=berry, numbers=none]
res = def (a, b) return a + b end (1, 2) # 3
\end{lstlisting}
在这个例程中,我们在函数字面值后面直接使用调用表达式来调用函数。这种用法很适合于那种只会在一个位置进行调用的函数。

还可以把匿名函数绑定到变量之后调用:
\begin{lstlisting}[language=berry, numbers=none]
add = def (a, b) return a + b end
res = add(1, 2) # 3
\end{lstlisting}
这种用法与具名函数的调用类似,本质上都是对绑定了函数值的变量执行调用。需要注意的是,对匿名函数进行递归调用会比较困难,除非你使用前向调用的技巧。

\subsection{形参和实参}

函数在调用时会使用实参来初始化形参。通常情况下,实参和形参数量相等且位置一一对应,不过Berry也允许实参数量不等于形参:如果实参数量多余形参则多出的实参会被丢弃,如果实参数量少于形参则会把余下的形参初始化为 \texttt{nil}。

参数传递的过程与赋值运算相似。对于 \texttt{nil}、\texttt{boolean} 和数值类型,参数传递是值传递,而其他类型是传递引用。对于实例这种可写的传引用类型,在被调函数中修改它们也会修改主调函数中的对象。下面的例子展示了这个特性:
\begin{lstlisting}[language=berry]
var l = [], i = 0
def func(a, b)
    a.append(1)
    b = 'string'
end
func(l, i)
print(l, i)     # [1] 0
\end{lstlisting}
可以看到,调用函数 \texttt{func} 之后变量 \texttt{l} 的值发生了变化,而变量 \texttt{i} 的值没有变化。

\subsection{函数和局部变量}

函数体本身是一个作用域,因此在函数中定义的变量都是局部变量(参考\ref{section::scope_life}节)。和直接嵌套的块不同,每一次调用函数都会为局部变量分配空间。局部变量的空间在栈中分配,并且分配信息是在编译期确定的,因此这个过程非常快。当函数中嵌套有多层作用域时,解释器会为依据局部变量最多的作用域嵌套链来分配栈空间,而不是依据函数中局部变量的总数。

\subsection{\texttt{return} 语句}

\texttt{return} 语句用于返回函数的结果,也就是函数的返回值。Berry中的所有函数都具有返回值,但是你可以在函数体中不使用任何 \texttt{return} 语句,此时解释器会生成一条默认的 \texttt{return} 语句以保证函数的返回。\texttt{return} 语句有两种写法:
\begin{algorithm}
    \texttt{return} \\
    \texttt{return }$\bm{expression}$
\end{algorithm}\vspace{-0.6em}\\
第一种写法是只写出 \texttt{return} 关键字而不写要返回的表达式,这种情况下返回默认的 \texttt{nil} 值。第二种写法是在 \texttt{return} 关键字后面跟随表达式$\bm{expression}$,此时会把该表达式的值作为函数的返回值。程序执行到 \texttt{return} 语句时,当前运行的函数会结束执行并返回到调用该函数的代码中继续运行。

当使用单独的关键字 \texttt{return} 作为函数的返回语句时,容易引起二义性的问题,此时建议在 \texttt{return} 后面加分号来防止出现错误:
\begin{lstlisting}[language=berry, numbers=none]
def func()
    return;
    x = 1
end
\end{lstlisting}
在这个例子里,\texttt{return} 语句后的 \texttt{x = 1} 语句不会得到执行,因此是多余的。如果避免这种冗余的代码,\texttt{return} 语句后面通常会跟随 \texttt{end}、\texttt{else} 或者 \texttt{elif} 等关键字,这种情况即使使用单独的 \texttt{return} 语句时也不用担心出现歧义。

\section{闭包}

\subsection{基础概念}

前面已经提到,函数在 Berry 中是第一类值,你可以在任何地方定义函数,也可以把函数作为参数或者返回值进行传递。在函数中定义另一个函数时,嵌套定义的函数可以访问任何外层函数的局部变量。我们把函数中使用的“外层函数的局部变量”称为函数的\textbf{自由变量},广义的自由变量也包括全局变量,但是在 Berry 中没有这项规则。

\textbf{闭包}是将函数和\textbf{环境}绑定的一种技术。环境是一个映射,它将函数的每个自由变量与一个值相关联。在实现上,闭包会将函数原型与自有变量关联存储。函数原型在编译期生成,而环境是一个运行时的概念,因此闭包也是在运行时动态生成的。每个闭包都是在生成时将函数原型和环境进行绑定,例如在下面的这个例子中:
\begin{lstlisting}[language=berry]
def func(i)     # The outer function
    def foo()   # The inner function (closure)
        print(i)
    end
    foo()
end
\end{lstlisting}
内层函数 \texttt{foo} 是一个闭包,它有一个自由变量 \texttt{i},该变量是外层函数 \texttt{func} 的一个参数。在生成闭包 \texttt{foo} 时会将它的函数原型和包含了自由变量 \texttt{i} 的环境绑定,当变量 \texttt{foo} 离开作用域以后闭包会被销毁。通常,内层函数会作为外层函数的返回值,例如:
\begin{lstlisting}[language=berry]
def func(i)         # The outer function
    return def ()   # Return a closure (anonymous function)
        print(i)
        i = i + 1
    end
end
\end{lstlisting}
这里返回的闭包是一个匿名函数。当闭包被外层函数返回以后,外层函数的局部变量会被销毁,闭包将不能直接访问原来外层函数中的变量。系统会在自由变量被销毁时将自由变量的值复制到环境中。这些自由变量的生命周期和闭包一致,并且只能由闭包访问。返回的函数或闭包并不会自动执行,因此我们要调用 \texttt{func} 函数返回的闭包:
\begin{lstlisting}[language=berry]
f = func(0)
f()
\end{lstlisting}
这段代码会输出 \texttt{0},如果我们继续调用闭包 \texttt{f},将会得到输出 \texttt{1}, \texttt{2}, \texttt{3}\ldots\ 这可能不是很好理解:变量 \texttt{i} 在函数 \texttt{func} 返回后被销毁,而作为闭包 \texttt{f} 的自由变量,\texttt{i} 会保存到闭包的环境中,因此每次调用 \texttt{f} 都会使 \texttt{i} 的值加 $1$(\texttt{func} 函数定义的第 4 行)。

\subsubsection{闭包的用途}

闭包有很多用途,这里介绍几种常见的用途:

\paragraph{惰性求值}

闭包在被调用前不会做任何事情。

\paragraph{函数的私有通信}

可以让一些闭包共用自由变量,这些自由变量仅对这些闭包可见,通过改变这些自由变量的值来进行函数间的通信。这样做可以避免使用外部变量。

\paragraph{生成多个函数}

有时候我们可能需要使用多个函数,这些函数可能只是一些变量的取值有所不同。我们可以实现一个函数,然后将这些不同的变量作为函数参数。更好的办法是通过一个工厂函数来返回闭包,并且把这些可能不同的变量作为闭包的自由变量,这样在调用函数时不必总是要写那些参数,而且可以生成任意数量的同类函数。

\paragraph{模拟私有成员}

有些语言支持在对象中使用私有成员,而 Berry 的类不支持私有成员。我们可以使用闭包的自由变量来模拟私有成员。这种用途并不是设计闭包的本意,不过在现在,这种对闭包的“误用”已经十分常见。

\paragraph{缓存结果}

如果有一个运行非常耗时的函数,那么每次调用它都会花费很多时间。我们可以缓存这个函数的结果,调用函数前先在缓存中查找,如果找到了就返回缓存值,否则调用函数并更新缓存值。我们可以利用闭包来保存缓存值,这样不会将它暴露到外层作用域,而缓存的结果也会被被保留(直到闭包销毁)。

\subsection{绑定自由变量}

如果多个闭包绑定了同一个自由变量,所有闭包将始终公用这个自由变量。例如:
\begin{lstlisting}[language=berry]
def func(i)     # The outer function
    return [    # Return a closure list
        def ()  # The closure #1
            print("closure 1 log:", i)
            i = i + 1
        end,
        def ()  # The closure #2
            print("closure 2 log:", i)
            i = i + 1
        end
    ]
end
\end{lstlisting}
这个例子里的 \texttt{func} 函数将两个闭包通过一个列表来返回,这两个闭包公用自由变量 \texttt{i}。如果我们调用这些闭包:
\begin{lstlisting}[language=berry]
f = func(0)
f[0]() # closure 1 log: 0
f[1]() # closure 2 log: 1
\end{lstlisting}
可以看到,我们调用闭包 \texttt{f[0]} 时更新了自由变量 \texttt{i},而这个变化影响了调用闭包 \texttt{f[1]} 的结果。这是由于如果一个自由变量被多个闭包使用,该自由变量也只有一份副本,所有的闭包都有一个对该自由变量实体的引用。因此,任何对该自由变量的修改都对所有使用了该自由变量的闭包可见。

同理,在外层函数的局部变量没有销毁之前,修改自由变量的值也会影响到闭包:
\begin{lstlisting}[language=berry]
def func()
    i = 0
    def foo()
        print(i)
    end
    i = 1
    return foo
end
\end{lstlisting}
在这个例子中,我们在外层函数 \texttt{func} 返回之前把变量 \texttt{i}(它是闭包 \texttt{foo} 的自由变量)的值从 \texttt{0} 改成了 \texttt{1},那么我们在此后调用闭包 \texttt{foo} 时自由变量 \texttt{i} 的值也是 \texttt{1}:
\begin{lstlisting}[language=berry]
func()() # 1
\end{lstlisting}

\subsection{在循环中创建闭包}

在循环体中构造闭包时,你可能不希望闭包的自由变量跟着循环循环变量变化。我们先来看一个在 \texttt{while} 循环中创建闭包的例子:
\begin{lstlisting}[language=berry]
def func()
    l = [] i = 0
    while (i <= 2)
        l.append(def () print(i) end)
        i = i + 1
    end
    return l
end
\end{lstlisting}
这个例子中,我们在循环中构造闭包,并把这个闭包放在一个 \texttt{list} 中。很明显,当循环结束后,变量 \texttt{i} 的值将是 \texttt{3},而列表 \texttt{l} 中所有的闭包也是使用这个变量的引用。如果我们执行 \texttt{func} 返回的闭包将得到相同的的结果:
\begin{lstlisting}[language=berry]
res = func()
res[0]() # 3
res[1]() # 3
res[2]() # 3
\end{lstlisting}
如果我们希望每个闭包都引用不同的自由变量,我们可以再定义一层函数,然后用函数的参数绑定当前的循环变量:
\begin{lstlisting}[language=berry]
def func()
    l = [] i = 0
    while (i <= 2)
        l.append(def (n)
            return def () print(n) end
        end (i))
        i = i + 1
    end
    return l
end
\end{lstlisting}
为了帮助理解这段看起来很难理解的代码,我们重点说明第 4 到第 6 行的代码:
\begin{lstlisting}[language=berry]
def (n)
    return def ()
        print(n)
    end
end (i)
\end{lstlisting}
这里实际上是定义了一个匿名函数并且立即调用它,这个临时的匿名函数的作用是将循环变量 \texttt{i} 的值绑定到它的参数 \texttt{n},而变量 \texttt{n} 也是我们所需闭包的自由变量,这样在每次循环时所构造的闭包绑定的自由变量都不同。现在我们将得到期望的输出:
\begin{lstlisting}[language=berry]
res = func()
res[0]() # 0
res[1]() # 1
res[2]() # 2
\end{lstlisting}
还有一些办法可以解决循环变量作为自由变量的问题。稍微简单一点的办法是在循环体中定义一个临时变量:
\begin{lstlisting}[language=berry]
def func()
    l = [] i = 0
    while (i <= 2)
        temp = i
        l.append(def () print(temp) end)
        i = i + 1
    end
    return l
end
\end{lstlisting}
这里的 \texttt{temp} 就是临时变量,该变量的作用域在循环体中,所以每次循环都会重新定义。我们还可以使用 \texttt{for} 语句来解决问题:
\begin{lstlisting}[language=berry]
def func()
    l = []
    for (i : 0 .. 2)
        l.append(def () print(i) end)
    end
    return l
end
\end{lstlisting}
这可能是最简洁的办法。\texttt{for} 语句的迭代变量会在每次循环中创建,其原理和与前一个办法类似。

\section{Lambda 表达式}

\textbf{Lambda 表达式}(lambda expression)表达式是一种特殊的匿名函数。Lambda 表达式由参数列表和函数体构成,但是形式和一般的函数不同:\vspace{-0.5em}
\begin{gather*}
    \texttt{/}\ args\ \texttt{->}\ expr
\end{gather*}
$\bm{args}$ 为参数列表,参数的数量可以是零到多个,多个参数之间使用逗号或者空格分开(不能同时混用);$\bm{expr}$ 为返回表达式,lambda 表达式将返回该表达式的值。Lambda 表达式适合实现功能非常简单的函数,例如判断两个数大小的 lambda 表达式为:
\begin{lstlisting}[language=berry, numbers=none]
/ a b -> a < b
\end{lstlisting}
这比同样功能的函数书写更简单。在一些通用排序算法中可能需要大量使用此类大小比较函数,使用 lambda 表达式可以简化代码并提高可读性。

与一般的函数相同,lambda 表达式可以构成闭包。Lambda 表达式的调用方式也和普通函数相同,如果使用类似匿名函数的立即调用方法:
\begin{lstlisting}[language=berry, numbers=none]
lambda = / a b -> a < b
result = lambda(1, 2)           # normal calling
result = (/ a b -> a < b)(1, 2) # direct calling
\end{lstlisting}
由于函数调用运算符的优先级比较高,进行直接调用时要在 lambda 表达式外加一对括号,这样会将其作为一个整体进行调用。

    \chapter{面向对象功能}

出于优化方面的考虑,Berry没有将简单类型作为对象,这些简单类型包括 \texttt{nil} 类型、数值类型、布尔类型和字符串类型。但是Berry提供了类来实现对象机制,在Berry的基本数据类型中,\texttt{list}、\texttt{map} 和 \texttt{range} 是类对象。一个对象是是包含数据和方法的集合,其中数据由一些变量来构成,而方法则是函数。对象的类型称为类(class),而对象的实体称为实例(instance)。

\section{类和实例}

\subsection{类的声明}

要使用一个类首先要进行声明。类的声明由关键字 \texttt{class} 开始,声明中要指定类的成员变量和方法,这是声明类的一个例子:
\begin{lstlisting}[language=berry, numbers=none]
class person
    var name, age
    def init(name, age)
        self.name = name
        self.age = age
    end
    def tostring()
        return 'name: ' + str(self.name) + ', age: ' + str(self.age)
    end
end
\end{lstlisting}

类的成员变量使用关键字 \texttt{var} 声明,而成员方法使用关键字 \texttt{def} 声明。目前Berry不支持在定义时初始化成员变量,因此成员变量的初始化工作应该由构造函数来完成。类的属性不能再声明完成后再做修改,因此类是一种只读对象\footnote{这种设计是为了保证在实现解释器的时候可以在C语言中静态构造类并使用 \texttt{const} 属性修饰以节省RAM}。

Berry 的类不支持访问限制,类的所有属性都对外部可见。在原生类中可以使用一些技巧使属性对 Berry 代码不可见(通常是让成员名字以``\text{.}''开头)。可以使用一些约定来限制对类中成员的访问,比如约定使用下划线开头的属性是私有属性,这种约定并不会在语法层面上有什么用,但是有利于代码的逻辑结构。

\subsection{实例化}

类本身只是一种抽象的描述。以汽车为例,我知道汽车的概念,而当我们真的要使用汽车的时候则需要真实的汽车。使用类的情况也类似,我们不会仅仅去使用这种抽象的描述,而是需要根据这种描述去生产出一个具体的对象。这个过程叫做\textbf{类的实例化},简称实例化,实例化产生的具体对象称为\textbf{实例}。类本身不具有数据,而实例化根据类所描述的信息生产一个实例并赋予实例具体的数据。

\subsection{方法和 \texttt{self} 参数}

类的方法本质上也是函数,与普通的函数不同,方法会隐式地传入一个 \texttt{self} 参数,且 \text{self} 总是作为第一个参数,该参数存储当前实例的引用。由于 \texttt{self} 参数的存在,方法的参数数量会比声明时定义的参数数量多一个。这里我们用一个简单的例子演示:
\begin{lstlisting}[language=berry, numbers=none]
class Test
    def method()
        return self
    end
end
object = Test()
print(object)
print(object.method())
\end{lstlisting}
这个例子中定义了一个 \texttt{Test} 类,它有一个 \texttt{method} 方法,该方法返回它的 \texttt{self} 参数。例程中的最后两行分别打印了 \texttt{Test} 类的实例 \texttt{object} 的值和使用 \texttt{method} 方法的返回值。该例子的运行结果为\footnote{由于实例对象是动态分配的,它们的内存地址是随机的,读者运行这段代码的结果可能与此处不同。}
\begin{lstlisting}[numbers=none]
<instance: 00E880D4>
<instance: 00E880D4>
\end{lstlisting}
可以看出,方法的 \texttt{self} 参数和使用实例的名字(例子中的 \texttt{object})都是表示同一个对象,它们都是实例的引用。使用 \texttt{self} 可以在方法中访问实例的成员或者属性。

\subsection{构造函数和析构函数}

\subsubsection{构造函数}

类的构造函数为 \texttt{init} 方法,构造函数会在类实例化的时候调用,因此构造函数一般用于成员的初始化工作,例如:
\begin{lstlisting}[language=berry, numbers=none]
class Test
    var a
    def init()
        self.a = 'this is a test'
    end
end
\end{lstlisting}
这个例子中的构造函数将 \texttt{Test} 类的 \texttt{a} 成员初始化为字符串 \texttt{'this is a test'}。如果我们实例化该类,就可以获得成员 \texttt{a} 的值:
\begin{lstlisting}[language=berry, numbers=none]
print(Test().a) # this is a test
\end{lstlisting}

\subsubsection{析构函数}

类的析构函数为 \texttt{deinit} 方法,析构函数会在实例被销毁时调用,析构函数一般用于完成一些清理工作。由于垃圾回收机制会自动释放无用对象的内存,因此不需要在析构函数中释放内存(也没有办法在析构函数中释放内存)。在大部分情况下都不需要使用析构函数,除非某个类要求在销毁时必须进行一定的处理,一个典型的例子是文件对象在销毁时必须关闭文件。

\section{类的继承}

Berry 只支持单继承,也就是类只能有一个基类,基类使用运算符 \texttt{:} 来声明:
\begin{lstlisting}[language=berry, numbers=none]
class Test : Base
    ...
end
\end{lstlisting}
这里 \texttt{Test} 类继承自 \texttt{Base} 类。子类会继承基类的所有方法和属性,同时你可以在子类中覆盖它们,这个机制被称为\textbf{重载}。通常情况下,我们只会重载方法,而不必重载属性。

Berry 类的继承机制比较简单,子类会包含基类的引用,实例对象也是类似。在实例化一个有基类的类时其实会生成多个对象,这些对象会根据继承关系链在一起,最后我们会拿到继承链最末端的实例对象。

\section{方法重载}

\textbf{重载}是指子类和基类使用同名的方法,而子类的方法将会覆盖基类方法的机制。准确地说成员变量也可以重载,但是这种重载没有任何意义。方法的重载分为普通方法重载以及运算符重载。

\subsection{普通方法重载}

\subsection{运算符重载}

\section{访问基类对象}

    \chapter {Libraries and Modules}

%==============================================================================
\section {Basic library}

There are some functions and classes that can be used directly in the standard library. They provide basic services for Berry programs, so they are also called basic libraries. The functions and classes in the basic library are visible in the global scope (belonging to the built-in scope), so they can be used anywhere. Do not define variables with the same name as the functions or classes in the base library. Doing so will make it impossible to reference the functions and classes in the base library.

\subsection {Built-in function}

%%%%%%%%%%%%%%%%%%%%%%%%%%%%%%%%%%%%%%%%%%%%%%%%%%%%%%%%%%%%%%%%%%%%%%%%%%%%%%%
\libtitle{\texttt{print} Functions}

\paragraph{usage}
\begin{lstlisting}[language=berry, numbers=none]
print(...)
\end{lstlisting}

\paragraph{Description}
This function prints the input parameters to the standard output device. The function can accept any type and any number of parameters. All types will print their value directly, and for an instance, this function will check whether the instance has a \texttt{tostring()} method, and if there is, print the return value of the instance calling the \texttt{tostring()} method, otherwise it will print the address of the instance.

\paragraph{example}
\begin{lstlisting}[language=berry, numbers=none]
print('Hello World!') # Hello World!
print([1, 2, '3']) # [1, 2, '3']
print(print) # <function: 0x561092293780>
\end{lstlisting}

%%%%%%%%%%%%%%%%%%%%%%%%%%%%%%%%%%%%%%%%%%%%%%%%%%%%%%%%%%%%%%%%%%%%%%%%%%%%%%%
\libtitle{\texttt{input} Function}

\paragraph{usage}
\begin{lstlisting}[language=berry, numbers=none]
input()
input(prompt)
\end{lstlisting}

\paragraph{Description}
\texttt{input} The function is used to input a line of character string from the standard input device. This function can use the \texttt{prompt} parameter as an input prompt, and the \texttt{prompt} parameter must be of string type.
After calling the \texttt{input} function, characters will be read from the keyboard buffer until a newline character is encountered.

\paragraph{example}
\begin{lstlisting}[language=berry, numbers=none]
input('please enter a string:') # please enter a string:
\end{lstlisting}
\texttt{input} The function does not return until the ``Enter'' key is pressed, so the program "stuck" is not an error.

%%%%%%%%%%%%%%%%%%%%%%%%%%%%%%%%%%%%%%%%%%%%%%%%%%%%%%%%%%%%%%%%%%%%%%%%%%%%%%%
\libtitle{\texttt{type} Function} \label{section::baselib_type}

\paragraph{usage}
\begin{lstlisting}[language=berry, numbers=none]
type(value)
\end{lstlisting}

\begin{itemize}
    \item \emph{value}: Input parameter (expect to get its type).
    \item \emph{return value}: A string describing the parameter type.
\end{itemize}

\paragraph{Description}
This function receives a parameter of any type and returns the type of the parameter. The return value is a string describing the type of the parameter. Table \ref{tab::type_return_list} shows the return values   corresponding to the main parameter types:
\begin{table}[htb]
    \centering
    \setlength{\tabcolsep}{6mm}
    \begin{tabular}{cc!{\vrule width 1pt}cc} \toprule
        \textbf{Parameter Type} & \textbf{return value} & \textbf{Parameter Type} & \textbf{return value} \\ \midrule
        Nil & \texttt{'nil'} & Integer & \texttt{'int'} \\
        Real & \texttt{'real'} & Boolean & \texttt{'bool'} \\
        Function & \texttt{'function'} & Class & \texttt{'class'} \\
        String & \texttt{'string'} & Instance & \texttt{'instance'} \\
        \bottomrule
    \end{tabular}
    \caption{Type name comparison table}
    \label{tab::type_return_list}
\end{table}

\paragraph{Example}
\begin{lstlisting}[language=berry, numbers=none]
type(0) #'int'
type(0.5) #'real'
type('hello') #'string'
type(print) #'function'
\end{lstlisting}

%%%%%%%%%%%%%%%%%%%%%%%%%%%%%%%%%%%%%%%%%%%%%%%%%%%%%%%%%%%%%%%%%%%%%%%%%%%%%%%
\libtitle{\texttt{classname} Functions}

\paragraph{usage}
\begin{lstlisting}[language=berry, numbers=none]
classname(object)
\end{lstlisting}

\paragraph{Description}
This function returns the class name (string) of the parameter. Therefore the parameter must be a class or instance, and other types of parameters will return \texttt{nil}.

\paragraph{Example}
\begin{lstlisting}[language=berry, numbers=none]
classname(list) #'list'
classname(list()) #'list'
classname({}) #'map'
classname(0) # nil
\end{lstlisting}

%%%%%%%%%%%%%%%%%%%%%%%%%%%%%%%%%%%%%%%%%%%%%%%%%%%%%%%%%%%%%%%%%%%%%%%%%%%%%%%
\libtitle{\texttt{classof} Functions}

\paragraph{usage}
\begin{lstlisting}[language=berry, numbers=none]
classof(object)
\end{lstlisting}

\paragraph{Description}
Returns the class of an instance object. The parameter \texttt{object} must be an instance. If the function is successfully called, it will return the class to which the instance belongs, otherwise it will return \texttt{nil}.

\paragraph{Example}
\begin{lstlisting}[language=berry, numbers=none]
classof(list) # nil
classof(list()) # <class: list>
classof({}) # <class: map>
classof(0) # nil
\end{lstlisting}

%%%%%%%%%%%%%%%%%%%%%%%%%%%%%%%%%%%%%%%%%%%%%%%%%%%%%%%%%%%%%%%%%%%%%%%%%%%%%%%
\libtitle{\texttt{str} Functions}\paragraph{usage}
\begin{lstlisting}[language=berry, numbers=none]
str(value)
\end{lstlisting}

\paragraph{Description}
This function converts the parameters into strings and returns. \texttt{str} Functions can accept any type of parameters and convert them. When the parameter type is an instance, it will check whether the instance has a \texttt{tostring()} method, if there is, the return value of the method will be used, otherwise the address of the instance will be converted into a string.

\paragraph{Example}
\begin{lstlisting}[language=berry, numbers=none]
str(0) # '0'
str(nil) #'nil'
str(list) #'list'
str([0, 1, 2]) #'[0, 1, 2]'
\end{lstlisting}

%%%%%%%%%%%%%%%%%%%%%%%%%%%%%%%%%%%%%%%%%%%%%%%%%%%%%%%%%%%%%%%%%%%%%%%%%%%%%%%
\libtitle{\texttt{number} Functions}

\paragraph{usage}
\begin{lstlisting}[language=berry, numbers=none]
number(value)
\end{lstlisting}

\paragraph{Description}
This function converts the input string or number into a numeric type and returns. If the input parameter is an integer or real number, it returns directly. If it is a character string, try to convert the character string to a numeric value in decimal format. The integer or real number will be automatically judged during the conversion. Other types return \texttt{nil}.

\paragraph{Example}
\begin{lstlisting}[language=berry, numbers=none]
number(5) # 5
number('45.6') # 45.6
number('50') # 50
number(list) # nil
\end{lstlisting}

%%%%%%%%%%%%%%%%%%%%%%%%%%%%%%%%%%%%%%%%%%%%%%%%%%%%%%%%%%%%%%%%%%%%%%%%%%%%%%%
\libtitle{\texttt{int} Functions}

\paragraph{usage}
\begin{lstlisting}[language=berry, numbers=none]
int(value)
\end{lstlisting}

\paragraph{Description}
This function converts the input string or number into an integer and returns it. If the input parameter is an integer, return directly, if it is a real number, discard the decimal part. If it is a string, try to convert the string to an integer in decimal. Other types return \texttt{nil}.

\paragraph{Example}
\begin{lstlisting}[language=berry, numbers=none]
int(5) # 5
int(45.6) # 45
int('50') # 50
int(list) # nil
\end{lstlisting}

%%%%%%%%%%%%%%%%%%%%%%%%%%%%%%%%%%%%%%%%%%%%%%%%%%%%%%%%%%%%%%%%%%%%%%%%%%%%%%%
\libtitle{\texttt{real} Functions}

\paragraph{usage}
\begin{lstlisting}[language=berry, numbers=none]
real(value)
\end{lstlisting}

\paragraph{Description}
This function converts the input string or number into a real number and returns. If the input parameter is a real number, it will return directly, if it is an integer, it will be converted to a real number. If it is a string, try to convert the string to a real number in decimal. Other types return \texttt{nil}.

\paragraph{Example}
\begin{lstlisting}[language=berry, numbers=none]
real(5) # 5, type(real(5)) →'real'
real(45.6) # 45.6
real('50.5') # 50.5
real(list) # nil
\end{lstlisting}

%%%%%%%%%%%%%%%%%%%%%%%%%%%%%%%%%%%%%%%%%%%%%%%%%%%%%%%%%%%%%%%%%%%%%%%%%%%%%%%
\libtitle{\texttt{length} Function}

\paragraph{usage}
\begin{lstlisting}[language=berry, numbers=none]
length(value)
\end{lstlisting}

\paragraph{Description}
This function returns the length of the input string. If the input parameter is not a string, 0 is returned. The length of the string is calculated in bytes.

\paragraph{Example}
\begin{lstlisting}[language=berry, numbers=none]
length(10) # 0
length('s') # 1
length('string') # 6
\end{lstlisting}

%%%%%%%%%%%%%%%%%%%%%%%%%%%%%%%%%%%%%%%%%%%%%%%%%%%%%%%%%%%%%%%%%%%%%%%%%%%%%%%
\libtitle{\texttt{super} Functions}

\paragraph{usage}
\begin{lstlisting}[language=berry, numbers=none]
super(object)
\end{lstlisting}

\paragraph{Description}
This function returns the parent object of the instance. When you instantiate a derived class, it will also instantiate its base class. The \texttt{super} function is required to access the instance of the base class (that is, the parent object).

\paragraph{Example}
\begin{lstlisting}[language=berry, numbers=none]
class mylist: list end
l = mylist() # classname(l) -->'mylist'
sl = super(l) # classname(sl) -->'list'
\end{lstlisting}

%%%%%%%%%%%%%%%%%%%%%%%%%%%%%%%%%%%%%%%%%%%%%%%%%%%%%%%%%%%%%%%%%%%%%%%%%%%%%%%
\libtitle{\texttt{assert} Function}\paragraph{usage}
\begin{lstlisting}[language=berry, numbers=none]
assert(expression)
assert(expression, message)
\end{lstlisting}

\paragraph{Description}
This function is used to implement the assertion function. \texttt{assert} The function accepts a parameter. When the value of the parameter is \texttt{false} or \texttt{nil}, the function will trigger an assertion error, otherwise the function will not have any effect. It should be noted that even if the parameter is a value equivalent to \texttt{false} in logical operations (for example, \texttt{0}), it will not trigger an assertion error. The parameter \texttt{message} is optional and must be a string. If this parameter is used, the text information given in \texttt{message} will be output when an assertion error occurs, otherwise the default ``\texttt{Assert Failed}'' message will be output.

\paragraph{Example}
\begin{lstlisting}[language=berry, numbers=none]
assert(false) # assert failed!
assert(nil) # assert failed!
assert() # assert failed!
assert(0) # assert failed!
assert(false,'user assert message.') # user assert message.
assert(true) # pass
\end{lstlisting}

%%%%%%%%%%%%%%%%%%%%%%%%%%%%%%%%%%%%%%%%%%%%%%%%%%%%%%%%%%%%%%%%%%%%%%%%%%%%%%%
\libtitle{\texttt{compile} Functions}

\paragraph{usage}
\begin{lstlisting}[language=berry, numbers=none]
compile(string)
compile(string,'string')
compile(filename,'file')
\end{lstlisting}

\paragraph{Description}
This function compiles the Berry source code into a function. The source code can be a string or a text file. \texttt{compile} The first parameter of the function is a string, and the second parameter is a string \texttt{'string'} or \texttt{'file'}. When the second parameter is \texttt{'string'} or there is no second parameter, the \texttt{compile} function will compile the first parameter as the source code. When the second parameter is \texttt{'file'}, the \texttt{compile} function will compile the file corresponding to the first parameter. If the compilation is successful, \texttt{compile} will return the compiled function, otherwise it will return \texttt{nil}.

\paragraph{Example}
\begin{lstlisting}[language=berry, numbers=none]
compile('print(\'Hello World!\')')() # Hello World!
compile('test.be','file')
\end{lstlisting}

%==============================================================================

\subsection{\texttt{list} Class}

\texttt{list} is a built-in type, which is a sequential storage container that supports subscript reading and writing. \texttt{list} Similar to arrays in other programming languages. Obtaining an instance of the \texttt{list} class can be constructed using a pair of square brackets: \texttt{[]} will generate an empty \texttt{list} instance, and \texttt{[expr, expr, ...]} will generate a \texttt{list} instance with several elements. It can also be instantiated by calling the \texttt{list} class: executing \texttt{list()} will get an empty \texttt{list} instance, and \texttt{list(expr, expr, ...)} will return an instance with several elements.

\libtitle{\texttt{list} Method (Constructor)}

Initialize the \texttt{list} container. This method can accept 0 to multiple parameters. The \texttt{list} instance generated when multiple parameters are passed will have these parameters as elements, and the arrangement order of the elements is consistent with the arrangement order of the parameters.

\libtitle{\texttt{tostring} Method}

Serialize the \texttt{list} instance to a string and return it. For example, the result of executing \texttt{[1, [], 1.5].tostring()} is \texttt{'[1, [], 1.5]'}. If the \texttt{list} container refers to itself, the corresponding position will use an ellipsis instead of the specific value:
\begin{lstlisting}[language=berry, numbers=none]
l = [1, 2]
l[0] = l
print(l) # [[...], 2]
\end{lstlisting}

\libtitle{\texttt{push} Method}

Append an element to the end of the \texttt{list} container. The prototype of this method is \texttt{push(value)}, the parameter \texttt{value} is the value to be appended, and the appended value is stored at the end of the \texttt{list} container. The append operation increases the number of elements in the \texttt{list} container by 1. You can append any type of value to the \texttt{list} instance.

\libtitle{\texttt{insert} Method}Insert an element at the specified position of the \texttt{list} container. The prototype of this method is \texttt{insert(index, value)}, the parameter \texttt{index} is the position to be inserted, and \texttt{value} is the value to be inserted. After inserting an element at the position \texttt{index}, all the elements that originally started from this position will move backward by one element. The insert operation increases the number of elements in the \texttt{list} container by 1. You can insert any type of value into the \texttt{list} container.

Suppose that the value of a \texttt{list} instance \texttt{l} is \texttt{[0, 1, 2]}, and we insert a string \texttt{'string'} at position 1, and we need to call \texttt{l.insert(1, 'string')}. Finally, the new \texttt{list} value is \texttt{[0, 'string', 1, 2]}.

If the number of elements in a \texttt{list} container is $S$, the value range of the insertion position is $\{i \in \mathbb{Z}: -S\leqslant i<S\}$. When the insertion position is positive, index backward from the head of the \texttt{list} container, otherwise index forward from the end of the \texttt{list} container.

\libtitle{\texttt{remove} Method}

Remove an element from the container. The prototype of this method is \texttt{remove(index)}, and the parameter \texttt{index} is the position of the element to be removed. After the element is removed, the element behind the removed element will move forward by one element, and the number of elements in the container will be reduced by 1. Like the \texttt{insert} method, the \texttt{remove} method can also use positive or negative indexes.

\libtitle{\texttt{item} Methods}

Get an element in the \texttt{list} container. The prototype of this method is \texttt{item(index)}, the parameter \texttt{index} is the index of the element to be obtained, and the return value of the method is the element at the index position. \texttt{list} The container supports multiple indexing methods:

\begin{itemize}
    \item Integer index: The index value can be a positive integer or a negative integer (the same as the index method in \texttt{insert}). At this time, the return value of \texttt{item} is the element at the index position. If the index position exceeds the number of elements in the container or is before the 0th element, the \texttt{item} method will return \texttt{nil}.
    \item \texttt{list} Index: Using a list of integers as an index, \texttt{item} returns a \texttt{list}, and each element in the return value \texttt{list} is an element corresponding to each integer index in the parameter \texttt{list}. The value of the expression \texttt{[3, 2, 1].item([0, 2])} is \texttt{[3, 1]}. If an element type in the parameter \texttt{list} is not an integer, then the value at that position in the return value \texttt{list} is \texttt{nil}.
    \item \texttt{range} Index: Using an integer range as an index, \texttt{item} returns a \texttt{list}. The returned value stores the indexed elements from \texttt{list} from the lower limit to the upper limit of the parameter \texttt{range}. If the index exceeds the index range of the indexed \texttt{list}, the return value \texttt{list} will use \texttt{nil} to fill the position beyond the index.
\end{itemize}

\libtitle{\texttt{setitem} Method}

Set the value of the specified position in the container. The prototype of this method is \texttt{setitem(index, value)}, \texttt{index} is the position of the element to be written, and \texttt{value} is the value to be written. \texttt{index} is the integer index value of the writing position. Index positions outside the index range of the container will cause \texttt{setitem} to fail to execute.

\libtitle{\texttt{size} Method}

Returns \texttt{list} the number of elements in the container, which is the length of the container. The prototype of this method is \texttt{size()}.

\libtitle{\texttt{resize} Method}

Reset \texttt{list} the length of the container. The prototype of this method is \texttt{resize(count)}, and the parameter \texttt{count} is the new length of the container. When using \texttt{resize} to increase the length of the container, the new element will be initialized to \texttt{nil}. Using \texttt{reszie} to reduce the length of the container will discard some elements at the end of the container. E.g:
\begin{lstlisting}[language=berry, numbers=none]
l = [1, 2, 3]
l.resize(5) # Expansion, l == [1, 2, 3, nil, nil]
l.resize(2) # Reduce, l == [1, 2]
\end{lstlisting}

\libtitle{\texttt{iter} Method}

Returns an iterator for traversing the current \texttt{list} container.

%==============================================================================
\subsection{\texttt{map} Class}

\texttt{map} Class is a built-in class type used to provide an unordered container of key-value pairs. Inside the Berry interpreter, \texttt{map} uses the Hash table to implement. You can use curly brace pairs to construct a \texttt{map} container. Using an empty curly brace pair \texttt{\{\}} will generate an empty \texttt{map} instance. If you need to construct a non-empty \texttt{map} instance, use a colon to separate the key and value, and use a semicolon to separate multiple key-value pairs. For example, \texttt{\{0: 1, 2: 3\}} has two key-value pairs $(0, 1)$ and $(2, 3)$. You can also get an empty \texttt{map} instance by calling the \texttt{map} class.

\libtitle{\texttt{map} Method (Constructor)}

Initialize the \texttt{map} container, this method does not accept parameters. Executing \texttt{map()} will get an empty \texttt{map} instance.

\libtitle{\texttt{tostring} Method}Serialize \texttt{map} as a string and return. The serialized string is similar to literal writing. For example, the result of executing \texttt{{'str': 1, 0: 2}} is \texttt{"{'str': 1, 0: 2}"}. If the \texttt{map} container refers to itself, the corresponding position will use an ellipsis instead of the specific value:
\begin{lstlisting}[language=berry, numbers=none]
m = {'map': nil,'text':'hello'}
m['map'] = m
print(m) # {'text':'hello','map': {...}}
\end{lstlisting}

\libtitle{\texttt{insert} Method}

Insert a key-value pair in the \texttt{map} container. The prototype of this method is \texttt{insert(key, value)}, the parameter \texttt{key} is the key to be inserted, and \texttt{value} is the value to be inserted. If the key \texttt{map} to be inserted exists in the container, the original key-value pair will be updated.

\libtitle{\texttt{remove} Method}

Remove a key-value pair from the \texttt{map} container. The prototype of this method is \texttt{remove(key)}, and the parameter \texttt{key} is the key of the key-value pair to be deleted.

\libtitle{\texttt{item} Method}

Get a value in the \texttt{map} container. The prototype of this method is \texttt{item(key)}, the parameter \texttt{key} is the key of the value to be obtained, and the return value of the method is the value corresponding to the key.

\libtitle{\texttt{setitem} Method}

Set the value corresponding to the specified key in the container. The prototype of this method is \texttt{setitem(key, value)}, \texttt{key} is the key of the key-value pair to be written, and \texttt{value} is the value to be written. If there is no key-value pair with the key \texttt{key} in the container, the \texttt{setitem} method will fail.

\libtitle{\texttt{size} Method}

Return \texttt{map} The number of key-value pairs of the container, which is the length of the container. The prototype of this method is \texttt{size()}.

%==============================================================================
\subsection{\texttt{range} Class}

\texttt{range} The class is used to represent an integer closed interval. Use the binary operator \texttt{..} to construct an instance of \texttt{range}. The left and right operands of the operator are required to be integers. For example, \texttt{0..10} means the integer interval $[0,10]\cap\mathbb{Z}$.

%==============================================================================
\section {Expansion Module}
%==============================================================================
\subsection {JSON Module}

JSON is a lightweight data exchange format. It is a subset of JavaScript. It uses a text format that is completely independent of the programming language to represent data. Berry provides a JSON module to provide support for JSON data. The JSON module only contains two functions \texttt{load} and \texttt{dump}, which are used to parse JSON strings and multiply Berry objects and serialize a Berry object into JSON text.

%%%%%%%%%%%%%%%%%%%%%%%%%%%%%%%%%%%%%%%%%%%%%%%%%%%%%%%%%%%%%%%%%%%%%%%%%%%%%%%
\libtitle{\texttt{load} Functions}

\paragraph{usage}
\begin{lstlisting}[language=berry, numbers=none]
load(text)
\end{lstlisting}

\paragraph{Description}
This function is used to convert the input JSON text into a Berry object and return it. The conversion rules are shown in Table \ref{tab::json2berry_rule}. If there is a syntax error in the JSON text, the function will return \texttt{nil}.
\begin{table}[htb]
    \centering
    \setlength{\tabcolsep}{18mm}
    \begin{tabular}{cc} \toprule
        \textbf{JSON type} & \textbf{Berry type} \\ \midrule
        \texttt{null} & \texttt{nil} \\
        \texttt{number} & \texttt{integer} or \texttt{real} \\
        \texttt{string} & \texttt{string} \\
        \texttt{array} & \texttt{list} \\
        \texttt{object} & \texttt{map} \\
        \bottomrule
    \end{tabular}
    \caption{JSON type to Berry type conversion rules}
    \label{tab::json2berry_rule}
\end{table}

\paragraph{Example}
\begin{lstlisting}[language=berry, numbers=none]
import json
json.load('0') # 0
json.load('[{"name": "liu", "age": 13}, 10.0]') # [{'name':'liu','age': 13}, 10]
\end{lstlisting}

%%%%%%%%%%%%%%%%%%%%%%%%%%%%%%%%%%%%%%%%%%%%%%%%%%%%%%%%%%%%%%%%%%%%%%%%%%%%%%%
\libtitle{\texttt{dump} Functions}

\paragraph{usage}
\begin{lstlisting}[language=berry, numbers=none]
dump(object, ['format'])
\end{lstlisting}

\paragraph{Description}
This function is used to serialize the Berry object into JSON text. The conversion rules for serialization are shown in Table \ref{tab::berry2json_rule}.
\begin{table}[htb]
    \centering
    \setlength{\tabcolsep}{18mm}
    \begin{tabular}{cc} \toprule
        \textbf{Berry type} & \textbf{JSON type} \\ \midrule
        \texttt{nil} & \texttt{null} \\
        \texttt{integer} & \texttt{number} \\
        \texttt{real} & \texttt{number} \\
        \texttt{list} & \texttt{array} \\
        \texttt{map} & \texttt{object} \\
        \texttt{map}Key of & \texttt{string} \\
        other & \texttt{string} \\
        \bottomrule
    \end{tabular}
    \caption{Berry type to JSON type conversion rules}
    \label{tab::berry2json_rule}
\end{table}

\paragraph{Example}
\begin{lstlisting}[language=berry, numbers=none]
import json
json.dump('string') #'"string"'
json.dump('string') #'"string"'
json.dump({0:'item 0','list': [0, 1, 2]}) #'{"0":"item 0","list":[0,1,2]}'
json.dump({0:'item 0','list': [0, 1, 2],'func': print},'format')
#-
{
    "0": "item 0",
    "list": [
        0,
        1,
        2
    ],
    "func": "<function: 00410310>"
}
-#
\end{lstlisting}

%==============================================================================
\subsection {Math Module}This module is used to provide support for mathematical functions, such as commonly used trigonometric functions and square root functions. To use the math module, first use the \texttt{import math} statement to import. All examples in this section assume that the module has been imported correctly.

%%%%%%%%%%%%%%%%%%%%%%%%%%%%%%%%%%%%%%%%%%%%%%%%%%%%%%%%%%%%%%%%%%%%%%%%%%%%%%%
\libtitle{\texttt{pi} Constants}

\paragraph{Description}
The approximate value of Pi $\pi$, a real number type, approximately equal to $3.141592654$.

\paragraph{Example}
\begin{lstlisting}[language=berry, numbers=none]
math.pi # 3.14159
\end{lstlisting}

%%%%%%%%%%%%%%%%%%%%%%%%%%%%%%%%%%%%%%%%%%%%%%%%%%%%%%%%%%%%%%%%%%%%%%%%%%%%%%%
\libtitle{\texttt{abs} Function}

\paragraph{usage}
\begin{lstlisting}[language=berry, numbers=none]
abs(value)
\end{lstlisting}

\paragraph{Description}
This function returns the absolute value of the parameter, which can be an integer or a real number. If there are no parameters, the function returns \texttt{0}, if there are multiple parameters, only the first parameter is processed. \texttt{abs} The return type of the function is a real number.

\paragraph{Example}
\begin{lstlisting}[language=berry, numbers=none]
math.abs(-1) # 1
math.abs(1.5) # 1.5
\end{lstlisting}

%%%%%%%%%%%%%%%%%%%%%%%%%%%%%%%%%%%%%%%%%%%%%%%%%%%%%%%%%%%%%%%%%%%%%%%%%%%%%%%
\libtitle{\texttt{ceil} Functions}

\paragraph{usage}
\begin{lstlisting}[language=berry, numbers=none]
ceil(value)
\end{lstlisting}

\paragraph{Description}
This function returns the rounded up value of the parameter, that is, the smallest integer value greater than or equal to the parameter. The parameter can be an integer or a real number. If there are no parameters, the function returns \texttt{0}, if there are multiple parameters, only the first parameter is processed. \texttt{ceil} The return type of the function is a real number.

\paragraph{Example}
\begin{lstlisting}[language=berry, numbers=none]
math.ceil(-1.2) # -1
math.ceil(1.5) # 2
\end{lstlisting}

%%%%%%%%%%%%%%%%%%%%%%%%%%%%%%%%%%%%%%%%%%%%%%%%%%%%%%%%%%%%%%%%%%%%%%%%%%%%%%%
\libtitle{\texttt{floor} Functions}

\paragraph{usage}
\begin{lstlisting}[language=berry, numbers=none]
floor(value)
\end{lstlisting}

\paragraph{Description}
This function returns the rounded down value of the parameter, which is not greater than the maximum integer value of the parameter. The parameter can be an integer or a real number. If there are no parameters, the function returns \texttt{0}, if there are multiple parameters, only the first parameter is processed. \texttt{floor} The return type of the function is a real number.

\paragraph{Example}
\begin{lstlisting}[language=berry, numbers=none]
math.floor(-1.2) # -2
math.floor(1.5) # 1
\end{lstlisting}

%%%%%%%%%%%%%%%%%%%%%%%%%%%%%%%%%%%%%%%%%%%%%%%%%%%%%%%%%%%%%%%%%%%%%%%%%%%%%%%
\libtitle{\texttt{sin} Function}

\paragraph{usage}
\begin{lstlisting}[language=berry, numbers=none]
sin(value)
\end{lstlisting}

\paragraph{Description}
This function returns the sine function value of the parameter. The parameter can be an integer or a real number, and the unit is radians. If there are no parameters, the function returns \texttt{0}, if there are multiple parameters, only the first parameter is processed. \texttt{sin} The return type of the function is a real number.

\paragraph{Example}
\begin{lstlisting}[language=berry, numbers=none]
math.sin(1) # 0.841471
math.sin(math.pi * 0.5) # 1
\end{lstlisting}

%%%%%%%%%%%%%%%%%%%%%%%%%%%%%%%%%%%%%%%%%%%%%%%%%%%%%%%%%%%%%%%%%%%%%%%%%%%%%%%
\libtitle{\texttt{cos} Functions}

\paragraph{usage}
\begin{lstlisting}[language=berry, numbers=none]
cos(value)
\end{lstlisting}

\paragraph{Description}
This function returns the value of the cosine function of the parameter. The parameter can be an integer or a real number in radians. If there are no parameters, the function returns \texttt{0}, if there are multiple parameters, only the first parameter is processed. \texttt{cos} The return type of the function is a real number.

\paragraph{Example}
\begin{lstlisting}[language=berry, numbers=none]
math.cos(1) # 0.540302
math.cos(math.pi) # -1
\end{lstlisting}

%%%%%%%%%%%%%%%%%%%%%%%%%%%%%%%%%%%%%%%%%%%%%%%%%%%%%%%%%%%%%%%%%%%%%%%%%%%%%%%
\libtitle{\texttt{tan} Functions}

\paragraph{usage}
\begin{lstlisting}[language=berry, numbers=none]
tan(value)
\end{lstlisting}

\paragraph{Description}
This function returns the value of the tangent function of the parameter. The parameter can be an integer or a real number, in radians. If there are no parameters, the function returns \texttt{0}, if there are multiple parameters, only the first parameter is processed. \texttt{tan} The return type of the function is a real number.

\paragraph{Example}
\begin{lstlisting}[language=berry, numbers=none]
math.tan(1) # 1.55741
math.tan(math.pi / 4) # 1
\end{lstlisting}%%%%%%%%%%%%%%%%%%%%%%%%%%%%%%%%%%%%%%%%%%%%%%%%%%%%%%%%%%%%%%%%%%%%%%%%%%%%%%%
\libtitle{\texttt{asin} Functions}

\paragraph{usage}
\begin{lstlisting}[language=berry, numbers=none]
asin(value)
\end{lstlisting}

\paragraph{Description}
This function returns the arc sine function value of the parameter. The parameter can be an integer or a real number. The value range is $[-1,1]$. If there are no parameters, the function returns \texttt{0}, if there are multiple parameters, only the first parameter is processed. \texttt{asin} The return type of the function is a real number and the unit is radians.

\paragraph{Example}
\begin{lstlisting}[language=berry, numbers=none]
math.asin(1) # 1.5708
math.asin(0.5) * 180 / math.pi # 30
\end{lstlisting}

%%%%%%%%%%%%%%%%%%%%%%%%%%%%%%%%%%%%%%%%%%%%%%%%%%%%%%%%%%%%%%%%%%%%%%%%%%%%%%%
\libtitle{\texttt{acos} Functions}

\paragraph{usage}
\begin{lstlisting}[language=berry, numbers=none]
acos(value)
\end{lstlisting}

\paragraph{Description}
This function returns the arc cosine function value of the parameter. The parameter can be an integer or a real number. The value range is $[-1,1]$. If there are no parameters, the function returns \texttt{0}, if there are multiple parameters, only the first parameter is processed. \texttt{acos} The return type of the function is a real number and the unit is radians.

\paragraph{Example}
\begin{lstlisting}[language=berry, numbers=none]
math.acos(1) # 0
math.acos(0) # 1.5708
\end{lstlisting}

%%%%%%%%%%%%%%%%%%%%%%%%%%%%%%%%%%%%%%%%%%%%%%%%%%%%%%%%%%%%%%%%%%%%%%%%%%%%%%%
\libtitle{\texttt{atan} Function}

\paragraph{usage}
\begin{lstlisting}[language=berry, numbers=none]
atan(value)
\end{lstlisting}

\paragraph{Description}
This function returns the arctangent function value of the parameter. The parameter can be an integer or a real number. The value range is $[-\infty,+\infty]$. If there are no parameters, the function returns \texttt{0}, if there are multiple parameters, only the first parameter is processed. \texttt{atan} The return type of the function is a real number and the unit is radians.

\paragraph{Example}
\begin{lstlisting}[language=berry, numbers=none]
math.atan(1) * 180 / math.pi # 45
\end{lstlisting}

%%%%%%%%%%%%%%%%%%%%%%%%%%%%%%%%%%%%%%%%%%%%%%%%%%%%%%%%%%%%%%%%%%%%%%%%%%%%%%%
\libtitle{\texttt{sinh} Function}

\paragraph{usage}
\begin{lstlisting}[language=berry, numbers=none]
sinh(value)
\end{lstlisting}

\paragraph{Description}
This function returns the hyperbolic sine function value of the parameter. If there are no parameters, the function returns \texttt{0}, if there are multiple parameters, only the first parameter is processed. \texttt{sinh} The return type of the function is a real number.

\paragraph{Example}
\begin{lstlisting}[language=berry, numbers=none]
math.sinh(1) # 1.1752
\end{lstlisting}

%%%%%%%%%%%%%%%%%%%%%%%%%%%%%%%%%%%%%%%%%%%%%%%%%%%%%%%%%%%%%%%%%%%%%%%%%%%%%%%
\libtitle{\texttt{cosh} Functions}

\paragraph{usage}
\begin{lstlisting}[language=berry, numbers=none]
cosh(value)
\end{lstlisting}

\paragraph{Description}
This function returns the hyperbolic cosine function value of the parameter. If there are no parameters, the function returns \texttt{0}, if there are multiple parameters, only the first parameter is processed. \texttt{cosh} The return type of the function is a real number.

\paragraph{Example}
\begin{lstlisting}[language=berry, numbers=none]
math.cosh(1) # 1.54308
\end{lstlisting}

%%%%%%%%%%%%%%%%%%%%%%%%%%%%%%%%%%%%%%%%%%%%%%%%%%%%%%%%%%%%%%%%%%%%%%%%%%%%%%%
\libtitle{\texttt{tanh} Functions}

\paragraph{usage}
\begin{lstlisting}[language=berry, numbers=none]
tanh(value)
\end{lstlisting}

\paragraph{Description}
This function returns the hyperbolic tangent function value of the parameter. If there are no parameters, the function returns \texttt{0}, if there are multiple parameters, only the first parameter is processed. \texttt{tanh} The return type of the function is a real number.

\paragraph{Example}
\begin{lstlisting}[language=berry, numbers=none]
math.tanh(1) # 0.761594
\end{lstlisting}

%%%%%%%%%%%%%%%%%%%%%%%%%%%%%%%%%%%%%%%%%%%%%%%%%%%%%%%%%%%%%%%%%%%%%%%%%%%%%%%
\libtitle{\texttt{sqrt} Function}

\paragraph{usage}
\begin{lstlisting}[language=berry, numbers=none]
sqrt(value)
\end{lstlisting}

\paragraph{Description}
This function returns the square root of the argument. The parameter of this function cannot be negative. If there are no parameters, the function returns \texttt{0}, if there are multiple parameters, only the first parameter is processed. \texttt{sqrt} The return type of the function is a real number.

\paragraph{Example}
\begin{lstlisting}[language=berry, numbers=none]
math.sqrt(2) # 1.41421
\end{lstlisting}

%%%%%%%%%%%%%%%%%%%%%%%%%%%%%%%%%%%%%%%%%%%%%%%%%%%%%%%%%%%%%%%%%%%%%%%%%%%%%%%
\libtitle{\texttt{exp} Functions}\paragraph{usage}
\begin{lstlisting}[language=berry, numbers=none]
exp(value)
\end{lstlisting}

\paragraph{Description}
This function returns the value of the parameter's exponential function based on the natural constant $e$. If there are no parameters, the function returns \texttt{0}, if there are multiple parameters, only the first parameter is processed. \texttt{exp} The return type of the function is a real number.

\paragraph{Example}
\begin{lstlisting}[language=berry, numbers=none]
math.exp(1) # 2.71828
\end{lstlisting}

%%%%%%%%%%%%%%%%%%%%%%%%%%%%%%%%%%%%%%%%%%%%%%%%%%%%%%%%%%%%%%%%%%%%%%%%%%%%%%%
\libtitle{\texttt{log} Functions}

\paragraph{usage}
\begin{lstlisting}[language=berry, numbers=none]
log(value)
\end{lstlisting}

\paragraph{Description}
This function returns the natural logarithm of the argument. The parameter must be a positive number. If there are no parameters, the function returns \texttt{0}, if there are multiple parameters, only the first parameter is processed. \texttt{log} The return type of the function is a real number.

\paragraph{Example}
\begin{lstlisting}[language=berry, numbers=none]
math.log(2.718282) # 1
\end{lstlisting}

%%%%%%%%%%%%%%%%%%%%%%%%%%%%%%%%%%%%%%%%%%%%%%%%%%%%%%%%%%%%%%%%%%%%%%%%%%%%%%%
\libtitle{\texttt{log10} Functions}

\paragraph{usage}
\begin{lstlisting}[language=berry, numbers=none]
log10(value)
\end{lstlisting}

\paragraph{Description}
This function returns the logarithm of the parameter to the base $10$. The parameter must be a positive number. If there are no parameters, the function returns \texttt{0}, if there are multiple parameters, only the first parameter is processed. \texttt{log10} The return type of the function is a real number.

\paragraph{Example}
\begin{lstlisting}[language=berry, numbers=none]
math.log10(10) # 1
\end{lstlisting}

%%%%%%%%%%%%%%%%%%%%%%%%%%%%%%%%%%%%%%%%%%%%%%%%%%%%%%%%%%%%%%%%%%%%%%%%%%%%%%%
\libtitle{\texttt{deg} Functions}

\paragraph{usage}
\begin{lstlisting}[language=berry, numbers=none]
deg(value)
\end{lstlisting}

\paragraph{Description}
This function is used to convert radians to angles. The unit of the parameter is radians. If there are no parameters, the function returns \texttt{0}, if there are multiple parameters, only the first parameter is processed. \texttt{deg} The return type of the function is a real number and the unit is an angle.

\paragraph{Example}
\begin{lstlisting}[language=berry, numbers=none]
math.deg(math.pi) # 180
\end{lstlisting}

%%%%%%%%%%%%%%%%%%%%%%%%%%%%%%%%%%%%%%%%%%%%%%%%%%%%%%%%%%%%%%%%%%%%%%%%%%%%%%%
\libtitle{\texttt{rad} Function}

\paragraph{usage}
\begin{lstlisting}[language=berry, numbers=none]
rad(value)
\end{lstlisting}

\paragraph{Description}
This function is used to convert angles to radians. The unit of the parameter is angle. If there are no parameters, the function returns \texttt{0}, if there are multiple parameters, only the first parameter is processed. \texttt{rad} The return type of the function is a real number and the unit is radians.

\paragraph{Example}
\begin{lstlisting}[language=berry, numbers=none]
math.rad(180) # 3.14159
\end{lstlisting}

%%%%%%%%%%%%%%%%%%%%%%%%%%%%%%%%%%%%%%%%%%%%%%%%%%%%%%%%%%%%%%%%%%%%%%%%%%%%%%%
\libtitle{\texttt{pow} Functions}

\paragraph{usage}
\begin{lstlisting}[language=berry, numbers=none]
pow(x, y)
\end{lstlisting}

\paragraph{Description}
The return value of this function is the result of the expression $x^y$, which is the parameter \texttt{x} to the \texttt{y} power. If the parameters are not complete, the function returns \texttt{0}, if there are extra parameters, only the first two parameters are processed. \texttt{pow} The return type of the function is a real number.

\paragraph{Example}
\begin{lstlisting}[language=berry, numbers=none]
math.pow(2, 3) # 8
\end{lstlisting}

%%%%%%%%%%%%%%%%%%%%%%%%%%%%%%%%%%%%%%%%%%%%%%%%%%%%%%%%%%%%%%%%%%%%%%%%%%%%%%%
\libtitle{\texttt{srand} Function}

\paragraph{usage}
\begin{lstlisting}[language=berry, numbers=none]
srand(value)
\end{lstlisting}

\paragraph{Description}
This function is used to set the seed of the random number generator. The type of the parameter should be an integer.

\paragraph{Example}
\begin{lstlisting}[language=berry, numbers=none]
math.srand(2)
\end{lstlisting}

%%%%%%%%%%%%%%%%%%%%%%%%%%%%%%%%%%%%%%%%%%%%%%%%%%%%%%%%%%%%%%%%%%%%%%%%%%%%%%%
\libtitle{\texttt{rand} Function}

\paragraph{usage}
\begin{lstlisting}[language=berry, numbers=none]
rand()
\end{lstlisting}

\paragraph{Description}
This function is used to get a random integer.

\paragraph{Example}
\begin{lstlisting}[language=berry, numbers=none]
math.rand()
\end{lstlisting}

\subsection {Time Module}

This module is used to provide time-related functions.

%%%%%%%%%%%%%%%%%%%%%%%%%%%%%%%%%%%%%%%%%%%%%%%%%%%%%%%%%%%%%%%%%%%%%%%%%%%%%%%
\libtitle{\texttt{time} Functions}\paragraph{prototype}
\begin{lstlisting}[language=berry, numbers=none]
time()
\end{lstlisting}

\paragraph{Description}
Returns the current timestamp. The timestamp is the time elapsed since Unix Epoch (1st January 1970 00:00:00 UTC), in seconds.

\libtitle{\texttt{dump} Functions}

\paragraph{prototype}
\begin{lstlisting}[language=berry, numbers=none]
dump(ts)
\end{lstlisting}

\paragraph{Description}
The input timestamp \texttt{ts} is converted into a time \texttt{map}, and the key-value correspondence is shown in Table \ref{tab::time_dump_map}.
\begin{table}[htb]
    \centering
    \setlength{\tabcolsep}{2mm}
    \begin{tabular}{cccccc} \toprule
        \textbf{key} & \textbf{value} & \textbf{key} & \textbf{value} & \textbf{key} & \textbf{value} \\ \midrule
        \texttt{'year'} & Year (from 1900) & \texttt{'month'} & Month (1-12) & \texttt{'day'} & Day (1-31) \\
        \texttt{'hour'} & Hour (0-23) & \texttt{'min'} & Points (0-59) & \texttt{'sec'} & Seconds (0-59) \\
        \texttt{'weekday'} & Week (1-7) \\
        \bottomrule
    \end{tabular}
    \caption{\texttt{time.dump} The key-value relationship of the function return value}
    \label{tab::time_dump_map}
\end{table}

\libtitle{\texttt{clock} Functions}

\paragraph{prototype}
\begin{lstlisting}[language=berry, numbers=none]
clock()
\end{lstlisting}

\paragraph{Description}

This function returns the elapsed time from the start of execution of the interpreter to when the function is called in seconds. The return value of this function is of type \texttt{real}, and its timing accuracy is determined by the specific platform.

\subsection{String Module}

The String module provides string processing functions.

\subsection{OS Module}

The OS module provides system-related functions, such as file and path-related functions. These functions are platform-related. Currently, Windows VC and POSIX style codes are implemented in the Berry interpreter. If it runs on other platforms, the functions in the OS module are not guaranteed to be provided.
    \chapter{API}

\section{API接口概述}


\end{document}
