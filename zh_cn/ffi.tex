\chapter{语言交互接口}

\textbf{语言交互接口}(Foreign Function Interface, FFI)是不同语言之间交互的接口。Berry提供了一套FFI来实现与C语言之间的交互,这套接口也非常容易在C++中使用。大部分的FFI接口是一些函数,它们的声明都放在\textit{berry.h}文件中。为了降低RAM的使用量,FFI还提供一套C编译期生成固定哈希表的机制,该机制必须要用到外部工具来生成C代码。

\section{基础}

FFI中最为重要的交互功能的应该是Berry代码与C函数相互调用的功能。为了实现两种语言互相调用对方的函数,我们要先了解Berry函数的参数传递机制。

\subsection{虚拟机}

与编译型语言不同,Berry语言不能直接在物理机器上运行,而是在特定的软件环境中运行,这个环境就是\textbf{虚拟机}(Virtual Machine, VM)。与真实的计算机相似,文本形式的源代码不能在虚拟机中执行,而要通过编译器将其转换为``字节码''才可以。Berry虚拟机定义为一个C结构体\texttt{bvm},这个结构体的内容对于FFI而言是不可见的。通过一些FFI函数,我们可以创建并初始化一个虚拟机。我们通过一个简单的例子来介绍虚拟机的使用:
\begin{lstlisting}[language=c, style=berry]
void berry_test(void)
{
    bvm *vm = be_vm_new();                     // construct a VM
    be_loadlibs(vm);                           // add libraries
    be_loadstring(vm, "print('Hello Berry')"); // compile test code
    be_pcall(vm, 0);                           // call function
    be_vm_delete(vm);                          // destroy the VM
}
\end{lstlisting}
这段代码给出了一个完整的使用虚拟机例子,首先,调用\texttt{be\_vm\_new}函数来构造一个新的虚拟机,之后所有的操作都在这个虚拟机对象中完成。默认情况下,虚拟机中不会加载任何模块或者函数,所以要调用\texttt{be\_loadlibs}来加载库,到此虚拟机的初始化工作就完成了。5到6行的作用是将一个字符串中的源代码编译成Berry函数然后调用。最后在第7行调用\texttt{be\_vm\_delete}函数销毁虚拟机。执行该函数将会在终端中得到一行输出:
\begin{lstlisting}[numbers=none]
Hello Berry
\end{lstlisting}

在所有的场景之下,虚拟机的构造、库加载以及销毁的流程都和上面例子中的第3行、第4行和第6行一样。更具需要,编译或者加载源代码的方式可能会有所不同,例如,对于文件形式的源代码,可以通过\texttt{be\_loadfile}函数来编译。源代码会被编译成一个Berry函数,且该函数会被存放在栈顶,通过调用FFI函数\texttt{be\_pcall}或\texttt{be\_call}既可执行该Berry函数。你还可以通过\texttt{be\_repl}函数来使用REPL,REPL的接口将在相关的章节中描述。

\subsection{虚拟栈}

Berry使用一个虚拟栈(virtual stack)和C语言编写的原生函数传递值,栈中每个元素都是一个Berry值。Berry代码调用原生函数时总会创建一个新的栈,并把所有的参数压入栈中。在C代码中也可以使用这个虚拟栈来存储数据,存储在栈中的值不会被垃圾回收器回收。

\begin{wrapfigure}{r}{0.5\textwidth}
\centering
\begin{tikzpicture}
    \begin{scope}[
    start chain=1 going right,start chain=2 going below,node distance=-0.15mm,minimum width=0.5cm,minimum height=0.6cm,font=\small\ttfamily
  ]
  \node [on chain=1] at (-1.5, -0.4) {\ldots};
  \foreach \x in {-1,...,-8} {
      \x, \node at (3.5+\x*0.5, 0.2) {\x};
  }
  \node [draw,on chain=1] {};
  \foreach \x in {1,...,8} {
      \x, \node (\x) [draw,on chain=1] {\x};
  }
  \foreach \x in {9,...,10} {
      \x, \node [draw,on chain=1] {};
  }
  \node [on chain=1] {\ldots};
  \node (base) at (-0.5, -1.2) {base};
  \node (top)  at (3, -1.2) {top};

  \draw[->] (base) -- (1);
  \draw[->] (top) -- (8);
  \end{scope}
\end{tikzpicture}
\caption{虚拟栈}
\label{fig::virtual_stack}
\end{wrapfigure}

Berry使用的堆栈如图\ref{fig::virtual_stack}所示,虚拟栈从左往右增长。Berry代码调用一个原生函数时会得到一个初始的堆栈,该堆栈第一个值的位置称为\textbf{栈底}(\texttt{base}),而最后一个值称为\textbf{栈顶}(\texttt{top}),函数只能访问堆栈中栈底到栈顶之间的值。栈底的位置是固定的,而栈顶的位置可以根据需要上涨或者下降。堆栈为空时,栈底指针\texttt{base}将会大于栈顶指针\texttt{top}。虚拟堆栈不严格遵循栈的操作规则:除了push和pop以外,虚拟栈还可以通过索引来访问,甚至可以在任意位置插入或者删除值。索引栈中元素的方法有两种:一种是以栈底为参考的\textbf{绝对索引},绝对索引值是从$1$开始的正整数;另一种是以栈顶为参考的\textbf{相对索引},相对索引值是从$-1$开始的负整数。以图\ref{fig::virtual_stack}为例,索引值$1,2\ldots 8$是绝对索引,元素的绝对索引就是该元素到栈底的距离。索引值$-1,-2\ldots -8$是相对索引,元素的相对索引值为该元素到栈顶距离的负数。如果一个索引值$index$有效,那么它所指的元素应该处于栈底到栈顶之间,也就是满足表达式$1\leq \mathrm{abs}(index)\leq top-base+1$。

\subsection{操作虚拟栈}

\subsubsection{索引和栈尺寸}

前面已经提到,可以使用两种索引方式访问虚拟栈,并且要保证索引值有效。同时,在很多时候也需要将新的值压入堆栈,此时须要程序员自己保证堆栈不会溢出。默认情况下,Berry保证有\texttt{BE\_STACK\_FREE\_MIN}个空间供原生函数使用,这个值可以在文件\textit{berry.h}中修改,它的默认值通常是$10$,在大多数情况下应该够用。如果确实需要扩充堆栈则,可以通过调用FFI函数\texttt{be\_stack\_require}来实现。该函数的原型为:
\begin{lstlisting}[language=c, style=berry, numbers=none]
void be_stack_require(bvm *vm, int count);
\end{lstlisting}
参数\texttt{count}为需要空间数量,当堆栈中的空位不足时将会扩展堆栈,否则这个函数什么也不做。

\textbf{警告}:如果发生了堆栈溢出,或者使用无效的索引访问堆栈都会导致程序崩溃。你可以打开调试开关\texttt{BE\_DEBUG},这会打开断言功能,可以在运行时得到一些调试信息,以便捕获堆栈溢出或者无效索引这类错误。

\subsubsection{从栈中获取值}

FFI中有一组函数用于从虚拟栈中获取值,这些函数通常会将栈中的值转换为C语言支持的简单值然后返回,以下是常用的从栈中获取值的FFI:
\begin{lstlisting}[language=c, style=berry, numbers=none]
bint be_toint(bvm *vm, int index);
breal be_toreal(bvm *vm, int index);
int be_tobool(bvm *vm, int index);
const char* be_tostring(bvm *vm, int index);
void* be_tocomptr(bvm *vm, int index);
\end{lstlisting}
这些函数的参数形式相同,只是返回值不同。前4个函数的作用和容易理解,就像它们的名字一样,\texttt{be\_toint}的功能是将虚拟栈中的值转换为C的整型值(\texttt{bint}通常是\texttt{int}类型的别名)并返回。最后一个函数\texttt{be\_tocomptr}的作用是从虚拟栈中取出一个通用类型的指针值,这个指针具体有何意义由C程序自己解释。

这些函数使用相同的方式去解释参数:参数\texttt{vm}是虚拟机实例的指针;\texttt{index}为待取出元素的索引。
